\documentclass[conference]{IEEEtran}

\usepackage[utf8]{inputenc}
\usepackage{graphicx}
\usepackage{cite}

\begin{document}

\title{Título del Documento en Formato IEEE}

\author{\IEEEauthorblockN{Nombre del Autor 1\IEEEauthorrefmark{1}, Nombre del Autor 2\IEEEauthorrefmark{2}}
\IEEEauthorblockA{\IEEEauthorrefmark{1}Departamento o Institución, Universidad o Compañía\\
Correo Electrónico: autor1@ejemplo.com}
\IEEEauthorblockA{\IEEEauthorrefmark{2}Otro Departamento o Institución\\
Correo Electrónico: autor2@ejemplo.com}
}

\maketitle

\begin{abstract}
El resumen debe proporcionar una visión general del contenido del artículo, destacando los puntos clave y los resultados principales en 150-250 palabras.
\end{abstract}

\begin{IEEEkeywords}
Palabra clave 1, Palabra clave 2, Palabra clave 3
\end{IEEEkeywords}

\section{Introducción}
Esta es la introducción del artículo. Aquí se presenta el contexto del trabajo, la motivación y los objetivos.

\section{Trabajo Relacionado}
En esta sección, se discuten investigaciones anteriores y cómo se relacionan con el presente trabajo.

\section{Metodología}
Describa aquí la metodología utilizada, incluyendo las técnicas, herramientas y procesos.

\section{Resultados}
Presente los resultados obtenidos. Utilice tablas, gráficos o figuras si es necesario.

\begin{table}[ht]
\caption{Ejemplo de una tabla en IEEE.}
\centering
\begin{tabular}{|c|c|c|}
\hline
Columna 1 & Columna 2 & Columna 3 \\ \hline
Dato 1    & Dato 2    & Dato 3    \\ \hline
Dato 4    & Dato 5    & Dato 6    \\ \hline
\end{tabular}
\end{table}

\section{Conclusión}
Resuma las conclusiones principales y sugiera direcciones futuras para la investigación.

\section*{Agradecimientos}
Los agradecimientos pueden incluir menciones a instituciones, organizaciones o individuos que ayudaron en el desarrollo del trabajo.

\bibliographystyle{IEEEtran}
\bibliography{bibliografia}

\end{document}
