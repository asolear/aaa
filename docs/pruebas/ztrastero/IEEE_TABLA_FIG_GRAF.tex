% Declarar los datos como una tabla simple

% % %%%%%%%%%%%%%%%%%%%%%%%%%%%%%%%%%%%%%%%%%%%%%%%
% % %%%%%%%%%%%%%%%%%%%%%%%%%%%%%%%%%%%%%%%%%%%%%%%
% % %%%%%%%%%%%%%%%%%%%%%%%%%%%%%%%%%%%%%%%%%%%%%%%
% % %%%%%%%%%%%%%%%%%%%%%%%%%%%%%%%%%%%%%%%%%%%%%%%
% % %%%%%%%%%%%%%%%%%%%%%%%%%%%%%%%%%%%%%%%%%%%%%%%

% \documentclass[conference]{IEEEtran}
\documentclass[conference,12pt]{IEEEtran}

\usepackage[pdftex,pdfencoding=auto]{hyperref}
\usepackage[utf8]{inputenc} % Allows UTF-8 input
\usepackage{amsmath} % For mathematical equations
\usepackage{lipsum} % Para texto falso
\usepackage{tikz} % Para diagramas
\usepackage{float}
\usepackage{ragged2e} % Para usar la justificación del texto
\usepackage[T1]{fontenc}    % Para una codificación de fuente adecuada
\usepackage{colortbl} % Paquete para colores en tablas
\usepackage{pgfplots}
\usepackage[spanish]{babel} % produce error con las flechas de los tikz
\usepackage{caption}
\usepackage{pgffor}  % Paquete para bucles



% % %%%%%%%%%%%%%%%%%%%%%%%%%%%%%%%%%%%%%%%%%%%%%%%
% \setlength{\spaceskip}{4.8pt}

% % Redefinir la numeración de secciones, subsecciones y subsubsecciones
% \renewcommand{\thesection}{\arabic{section}} % 1, 2, 3...
% \renewcommand{\thesubsection}{\thesection.\arabic{subsection}} % 1.1, 1.2...
% \renewcommand{\thesubsubsection}{\thesubsection.\arabic{subsubsection}} % 1.1.1, 1.1.2...

% \let\OldTextField\TextField
% \renewcommand{\TextField}[2][]{%
%   \raisebox{-0.3ex}{\OldTextField[height=.85em,  bordercolor={1 1 1}, backgroundcolor={1 1 1},#1]{#2}}%
% }

% \renewcommand{\baselinestretch}{1.5}  % 1.5 es el valor estándar, pero puedes aumentarlo a 2, 2.5, etc.
% \renewcommand{\rmdefault}{phv}  % Cambia la fuente de texto a Helvetica
% \fontsize{12}{15}\selectfont  % Establece el tamaño de la fuente y la altura de línea
% \onecolumn
% %%%%%%%%%%%%%%%%%%%%%%%%%%%%%%%%%%%%%%%%%%%%%%%
\title{titulo}
\author{}
\date{}
\begin{document}
\justifying
\begin{Form}
\maketitle
\tableofcontents
% \newpage
% % %%%%%%%%%%%%%%%%%%%%%%%%%%%%%%%%%%%%%%%%%%%%%%%
% % %%%%%%%%%%%%%%%%%%%%%%%%%%%%%%%%%%%%%%%%%%%%%%%
% % %%%%%%%%%%%%%%%%%%%%%%%%%%%%%%%%%%%%%%%%%%%%%%%
% % %%%%%%%%%%%%%%%%%%%%%%%%%%%%%%%%%%%%%%%%%%%%%%%



\begin{abstract}
Esteaa documento presenta ejemplos de tabla, gráfico de barras y un esquema básico de una caldera diseñado con TikZ.
\end{abstract}

\section{Introducción}
Se presentan ejemplos de cómo incluir tablas, gráficos y diagramas en documentos IEEE.
% Datos almacenados
\pgfplotstableread[col sep=comma]{
    x,y
    1,.2
    2,.3
    3,.5
    4,.7
    5,.11
}\misdatos

% Gráfico
% Crear el gráfico
\begin{tikzpicture}[scale=.63]
    \begin{axis}[
        xlabel={Eje X},
        ylabel={Eje Y},
        grid=both,
    ]
        \addplot+[ybar] table[x=x, y=y] {\misdatos};
        \addplot+ {1/sqrt(2*pi)*exp(-x^2/2)};
    \end{axis}
\end{tikzpicture}




\section{Tabla de Datos}
\begin{table}[h]
    \centering
    \caption{Datos de Ejemplo}
    \begin{tabular}{@{}llc@{}}
        \toprule
        \textbf{Categoría} & \textbf{Descripción} & \textbf{Valor} \\ \midrule
        A & Primer dato  & 10 \\
        B & Segundo dato & 20 \\
        C & Tercer dato  & 30 \\
        \bottomrule
    \end{tabular}
    \label{tab:example}
\end{table}

\section{Gráfico de Barras}
\begin{figure}[h]
    \centering
    \begin{tikzpicture}
        \begin{axis}[
            width=\linewidth,
            height=5cm,
            ybar,
            bar width=0.5cm,
            ylabel={Valor},
            symbolic x coords={A, B, C},
            xtick=data,
            nodes near coords,
            nodes near coords align={vertical},
            ymin=0,
        ]
        \addplot coordinates {(A,10) (B,20) (C,30)};
        \end{axis}
    \end{tikzpicture}
    \caption{Gráfico de barras de los datos.}
    \label{fig:bargraph}
\end{figure}

\section{Esquema de una Caldera}
\begin{figure}[h]
    \centering
    \begin{tikzpicture}[scale=1]
        % Cuerpo de la caldera
        \draw[thick] (0,0) rectangle (4,6);
        \node[align=center] at (2,3) {Cámara\\ de\\ Combustión};

        % Tuberías
        \draw[thick] (-1,4) -- (0,4); % Entrada
        \draw[thick] (4,5) -- (5,5); % Salida

        % Flechas
        \draw[->,thick] (-1,4) -- (-0.5,4) node[midway, above] {Entrada};
        \draw[->,thick] (4.5,5) -- (5,5) node[midway, above] {Salida};

        % Indicador de presión
        \draw[thick] (2,6) -- (2,7);
        \node[circle, draw, fill=white, minimum size=1cm] at (2,7.5) {};
        \node at (2,7.5) {P};

        % Base de la caldera
        \draw[thick] (0,-0.5) -- (4,-0.5);
        \draw[thick] (1,-0.5) -- (1,-1);
        \draw[thick] (3,-0.5) -- (3,-1);
    \end{tikzpicture}
    \caption{Esquema básico de una caldera.}
    \label{fig:boiler}
\end{figure}

\section{Conclusión}
Se ha demostrado cómo incluir elementos visuales en un documento IEEE.

\end{document}
