\newcommand{\resumen}{  El texto del resumen debe estar de acuerdo con la Norma UNE 50103. En síntesis, debe ser tan informativo como lo
permita la naturaleza del documento, para que los lectores puedan decidir si es necesario leer el documento completo;
debe definir el objetivo, métodos, resultados y conclusiones presentadas en el documento original, bien en ese orden,
o destacando inicialmente los resultados y conclusiones; debe constituir un texto completo, para que sea inteligible sin
necesidad de referirse al documento. Debe ser conciso sin ser oscuro, reteniendo la información básica y el carácter
del documento original. Los resúmenes de la mayoría de los informes deben tener menos de 250 palabras y en ningún
caso más de 500; Deben estar escritos en un solo párrafo; emplear normalmente frases completas, verbos en forma
activa y con tercera persona. No se deben utilizar figuras y símbolos, tales como tablas cortas y fórmulas, más que
cuando no haya ninguna alternativa aceptable.}
\title{Your Title Here}
\author{Your Name Here}
\date{\today}  % or specify a date, e.g., \date{December 9, 2024}


\input{../../../assets/settings/newcommand.tex}

\documentclass[a4paper,12pt]{article}

\usepackage[utf8]{inputenc}
\usepackage[spanish]{babel}
% hay que usar para que no den error las flechas de draw, pero antes del tikz
% \shorthandoff{<} % Desactiva el uso de < como comando
% \shorthandoff{>} % Desactiva el uso de < como comando
\usepackage{graphicx}
\usepackage{titlesec}
\usepackage{geometry}
\usepackage{fancyhdr}
\usepackage{setspace}
\usepackage{tikz} % Para diagramas
\usepackage{amsmath} % For mathematical equations
\usepackage{colortbl} % Paquete para colores en tablas
\usepackage{pgfplots}
\usepackage{caption}
\usepackage{pgffor}  % Paquete para bucles
\usepackage{fancyhdr} % Para personalizar encabezados y pies de página
\usepackage{pgfplotstable}
\usepackage{booktabs}
\usepackage[T1]{fontenc}    % Soporte para caracteres con acentos
\usepackage{textcomp}       % Soporte adicional para símbolos
\usepackage{circuitikz}
\usepackage{multicol}
\usepackage{ifthen}

% 
\usepackage[pdftex,pdfencoding=auto]{hyperref}
% \let\OldTextField\TextField
% \renewcommand{\TextField}[2][]{%
%   \raisebox{-0.1ex}{\OldTextField[height=.95em,  bordercolor={1 1 1}, backgroundcolor={1 1 1},#1]{#2}}%
% }
\geometry{left=3cm, right=2.5cm, top=3cm, bottom=2.5cm}
\titleformat{\section}{\normalfont\Large\bfseries}{\thesection}{1em}{}
\titleformat{\subsection}{\normalfont\large\bfseries}{\thesubsection}{1em}{}
\pagestyle{fancy}
\fancyhf{}
\fancyhead[L]{}
\fancyfoot[C]{\thepage}
\setstretch{1.5}

\begin{document}
\begin{Form}



\graphicspath{{../../../assets/imgs}}



\maketitle



\vspace{4cm}
    Codigo de documento\hspace{3cm} Fecha de emisión\\
    \TextField[multiline=true, bordercolor=gray!30,name=beee,width=0.499\textwidth,height=.5 cm,default=]{}
    \TextField[multiline=true, bordercolor=gray!30,name=beee,width=0.499\textwidth,height=.5cm,default=]{}
    Promotor\hspace{6cm} Ingenieria\\
    \TextField[multiline=true, bordercolor=gray!30,name=beee,width=0.499\textwidth,height=1.5 cm,default=]{}
    \TextField[multiline=true, bordercolor=gray!30,name=beee,width=0.499\textwidth,height=1.5cm,default=]{}
    % Titulo y subtitulo  \\
    % \TextField[multiline=true, bordercolor=gray!30,name=beee,width=1\textwidth,height=1.5 cm,default=]{}
    Autor  \\
    \TextField[multiline=true, bordercolor=gray!30,name=beee,width=1\textwidth,height=1 cm,default=]{}
    Resumen  \\
    \TextField[multiline=true, bordercolor=gray!30,name=beee,width=1\textwidth,height=4 cm,default=\resumen]{}
    Palabras clave  \\
    \TextField[multiline=true, bordercolor=gray!30,name=beee,width=1\textwidth,height=1.5 cm,default=]{}
    
\newpage
\tableofcontents

% \section*{Glosario de términos}

% \begin{description}
%   \item[LaTeX] Sistema de preparación de documentos basado en texto, utilizado para crear documentos de alta calidad tipográfica.
%   \item[PDF] Formato de archivo para documentos que preserva el formato y es independiente del software, hardware o sistema operativo utilizado.
%   \item[UNE] Normas técnicas desarrolladas por la Asociación Española de Normalización y Certificación.
%   \item[HTML] Lenguaje de marcado utilizado para la creación de páginas web.
%   \item[CSS] Lenguaje utilizado para describir la presentación de un documento escrito en HTML o XML.
%   \item[XML] Lenguaje de marcado que define reglas para la codificación de documentos en un formato legible por humanos y máquinas.
% \end{description}


\newpage

\section{Introducción}
En esta sección se describe el propósito del documento y su alcance. También se puede incluir información general y antecedentes necesarios para entender el contenido.

\section{Objeto y Campo de Aplicación}
Definir de forma clara y concisa el propósito principal del documento y los límites de su aplicación.

\section{Normas y Referencias Aplicables}
Enumerar las normas, reglamentos y documentos de referencia utilizados.


\section{Metodología}
Describir el método o procedimiento seguido. Esto debe estar en concordancia con los estándares aplicables.

\section{Resultados}
Exponer los resultados obtenidos. Utiliza tablas o gráficos si es necesario.

\section{Conclusiones y Recomendaciones}
Presentar las conclusiones principales y sugerir posibles acciones o estudios futuros.

\section{Referencias}
Libro: PETTERSEN, Sverre. Introduction to Meteorology. New York, MacGraw Hill, 1941: pp. 200-210.


\appendix
\section{Anexos}
En esta sección se incluyen tablas, gráficos, cálculos o documentos adicionales que complementan el contenido principal.

\subsection{Ilustraciones o tablas suplementarias.}
\pgfplotstableread[col sep=comma]{
    x,y
    1,.2
    2,.3
    3,.5
    4,.7
    5,.11
}\misdatos

\begin{figure}[ht]
    \centering
    \begin{tikzpicture}[scale=.63]
        \begin{axis}[
            xlabel={Eje X},
            ylabel={Eje Y},
            grid=both,
            xtick=\empty, % No mostrar los valores del eje X
            ytick=\empty  % No mostrar los valores del eje Y
        ]
            \addplot+[ybar] table[x=x, y=y] {\misdatos};
            \addplot+ {1/sqrt(2*pi)*exp(-x^2/2)};
                  % Spline usando to[out=angle1, in=angle2]
        \end{axis}
    \end{tikzpicture}
    \caption{Una figura de ejemplo} % Título de la figura
\end{figure}


\pgfplotstableread[col sep=comma]{
    Nombre dd dasdAS Asdas,Color,Forma
    Manzana,Rojo,{\TextField[name=Tecnico,width=2cm,default=Juan Pérez]{}}
    Plátano,Amarillo,Alargada
    Uva,Morado,Pequeña
    Limón,Verde,Ovalada
    Naranja,Naranja,Redonda
}\datosNoNumericos

\begin{table}[ht]
    \centering
    \caption{Una Tabla de ejemplo}
    \pgfplotstabletypeset[
        col sep=comma,
        every head row/.style={before row=\toprule, after row=\midrule},
        every last row/.style={after row=\bottomrule},
        string type % Indica que los datos son cadenas de texto
    ]{\datosNoNumericos}
\end{table}

\begin{figure}[h]
    \centering
    \includegraphics[width=0.8\linewidth]{fvresidencial.jpg} % Ajusta la ruta y el tamaño según tus necesidades
    \caption{Emplazamiento geográfico.}
    \label{fig:etiqueta}
\end{figure}

\subsection{Descripción de equipos, técnicas o programas de ordenador..}







\newpage
\section*{Bibliografía}
\begin{itemize}
    \item \href{https://www.ree.es/es/clientes/generador/gestion-medidas-electricas/consulta-perfiles-de-consumo}{REE. Consulta los perfiles de consumo (TBD)}
    \item Norma UNE XXXX: Año. Título de la norma.
\end{itemize}





\end{Form}
\end{document}
