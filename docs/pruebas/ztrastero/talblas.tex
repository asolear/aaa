\documentclass{article}
\usepackage{pgfplots}
\pgfplotsset{compat=1.18}

\begin{document}

\pgfplotstableread{
    x   y
    1   2
    2   3
    3   5
    4   7
}\datatable

% Graficar los datos
\begin{tikzpicture}
    \begin{axis}[
        xlabel={Eje X},
        ylabel={Eje Y},
        title={Ejemplo de pgfplots con tabla},
        grid=major
    ]
        \addplot table[x=x, y=y] {\datatable};
    \end{axis}
\end{tikzpicture}


% Datos no numéricos
\pgfplotstableread[col sep=comma]{
    Característica, Este-Oeste, Incl. Óptima, Horizontal, Sur con Incl. Óptima
    Producción de Energía, Media, Alta, Baja, Muy Alta
    Uso del Espacio, Alto, Medio, Alto, Medio
    Sombra, Baja, Media, N/A, Alta
    Facilidad de Instalación, Media, Baja, Alta, Baja
    Coste de Instalación, Alta, Baja, Alta, Media
    Resistencia al Viento, Media, Alta, Media, Media
    Mantenimiento, Media, Media, Alta, Media
    Distribución de Producción, Uniforme en día, Pico al mediodía, Baja, Pico al mediodía
    Estética e Integración, Media, Alta, Baja, Media
}\datosNoNumericos

% Tabla con datos no numéricos
\begin{table}[ht]
    \centering
    \caption{Comparativa de Alternativas para la Disposición de Paneles Solares en la Cubierta}
    \pgfplotstabletypeset[
        col sep=comma,
        every head row/.style={before row=\toprule, after row=\midrule},
        every last row/.style={after row=\bottomrule},
        string type % Indica que los datos son cadenas de texto
    ]{\datosNoNumericos}
\end{table}

\end{document}