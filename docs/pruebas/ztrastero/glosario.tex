\documentclass[a4paper,12pt]{article}
\usepackage[utf8]{inputenc}
\usepackage[spanish]{babel}
\usepackage{glossaries}

% Configuración del glosario
\makeglossaries

% Definición de términos
\newglossaryentry{une}{
    name=UNE,
    description={Norma Española de carácter técnico}
}
\newglossaryentry{latex}{
    name=LaTeX,
    description={Sistema de preparación de documentos basado en TeX}
}
\newglossaryentry{pdf}{
    name=PDF,
    description={Formato de Documento Portátil (Portable Document Format)}
}

\begin{document}

\title{Documento Ejemplo según Norma UNE}
\author{Autor del Documento}
\date{\today}
\maketitle

\tableofcontents

\section*{Glosario de signos, símbolos, unidades, abreviaturas, acrónimos o términos}
\glsaddall % Incluye todos los términos definidos
\printglossary % Imprime el glosario

NO FUNCIONA

\section*{Glosario de signos, símbolos, unidades, abreviaturas, acrónimos o términos}

\begin{description}
  \item[LaTeX] Sistema de preparación de documentos basado en texto, utilizado para crear documentos de alta calidad tipográfica.
  \item[PDF] Formato de archivo para documentos que preserva el formato y es independiente del software, hardware o sistema operativo utilizado.
  \item[UNE] Normas técnicas desarrolladas por la Asociación Española de Normalización y Certificación.
  \item[HTML] Lenguaje de marcado utilizado para la creación de páginas web.
  \item[CSS] Lenguaje utilizado para describir la presentación de un documento escrito en HTML o XML.
  \item[XML] Lenguaje de marcado que define reglas para la codificación de documentos en un formato legible por humanos y máquinas.
\end{description}
\section{Introducción}
En este documento se explican conceptos como \gls{une}, \gls{latex} y \gls{pdf}.

\end{document}
