\documentclass[tikz,border=2mm]{standalone}
\usepackage{amsmath}
\usepackage{colortbl} % Paquete para colores en tablas
\usepackage{pgfplots}
\usepackage{caption}
\usepackage{pgffor}  % Paquete para bucles
\usepackage{fancyhdr} % Para personalizar encabezados y pies de página
\usepackage{pgfplotstable}
\usepackage{booktabs}
\usepackage[T1]{fontenc}    % Soporte para caracteres con acentos
\usepackage{textcomp}       % Soporte adicional para símbolos
\usepackage{circuitikz}
\usepackage[spanish,safe]{babel}
\usepackage{graphicx}
\usepackage{titlesec}
\usepackage{geometry}
\usepackage{fancyhdr}
\usepackage[pdftex,pdfencoding=auto]{hyperref}
\usepackage{setspace}
\usepackage{tikz} % Para diagramas

\begin{document}

\shorthandoff{<} % Desactiva el uso de < como comando
\shorthandoff{>} % Desactiva el uso de < como comando

\begin{tikzpicture}
    % Dibujar el rectángulo
    \draw[thick] (0,0) rectangle (5,3);

    % Etiquetas de las cotas (horizontal)
    \draw[<->, thick] (0,-0.5) -- node[below]{5 cm} (5,-0.5);

    % Etiquetas de las cotas (vertical)
    \draw[<->, thick] (-0.5,0) -- node[left]{3 cm} (-0.5,3);

    % Etiqueta interna (diagonal)
    \draw[<->, dashed] (0,0) -- node[above, sloped]{Diagonal: $\sqrt{5^2 + 3^2}$} (5,3);
    
    % Agregar nombres de puntos (opcional)
    \node[below left] at (0,0) {A};
    \node[below right] at (5,0) {B};
    \node[above right] at (5,3) {C};
    \node[above left] at (0,3) {D};
\end{tikzpicture}

\end{document}