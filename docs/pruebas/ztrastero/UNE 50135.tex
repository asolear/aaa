\title{Estudio Técnico\\ \textbf{Mejora del Aislamiento Térmico de los Invernaderos mediante Pantallas Térmicas}}
% Declarar los datos como una tabla simple
\newcommand{\resumen}{ 
\begin{abstract}
fasdfsad
\end{abstract}  
}

% Declarar los datos como una tabla simple
% % %%%%%%%%%%%%%%%%%%%%%%%%%%%%%%%%%%%%%%%%%%%%%%%
% % %%%%%%%%%%%%%%%%%%%%%%%%%%%%%%%%%%%%%%%%%%%%%%%
% % %%%%%%%%%%%%%%%%%%%%%%%%%%%%%%%%%%%%%%%%%%%%%%%
% % %%%%%%%%%%%%%%%%%%%%%%%%%%%%%%%%%%%%%%%%%%%%%%%
% % %%%%%%%%%%%%%%%%%%%%%%%%%%%%%%%%%%%%%%%%%%%%%%%
\documentclass[conference,12pt]{IEEEtran}
\usepackage[pdftex,pdfencoding=auto]{hyperref}
\usepackage[utf8]{inputenc} % Allows UTF-8 input
\usepackage{lipsum} % Para texto falso
\usepackage{float}
\usepackage{ragged2e} % Para usar la justificación del texto
\usepackage[T1]{fontenc}    % Para una codificación de fuente adecuada

\usepackage{tikz} % Para diagramas
\usepackage{amsmath} % For mathematical equations
\usepackage{colortbl} % Paquete para colores en tablas
\usepackage{pgfplots}
\usepackage[spanish]{babel} % produce error con las flechas de los tikz
\usepackage{caption}
\usepackage{pgffor}  % Paquete para bucles
\usepackage{fancyhdr} % Para personalizar encabezados y pies de página
\usepackage{pgfplotstable}
\usepackage{booktabs}

% %%%%%%%%%%%%%%%%%%%%%%%%%%%%%%%%%%%%%%%%%%%%%%%
\addto\captionsspanish{\renewcommand{\tablename}{Tabla}}
\addto\captionsspanish{\renewcommand{\listtablename}{Índice de Tablas}} % Cambiar título a "Índice de Tablas"


\setlength{\spaceskip}{4.8pt}
% Redefinir la numeración de secciones, subsecciones y subsubsecciones
\renewcommand{\thesection}{\arabic{section}} % 1, 2, 3...
\renewcommand{\thesubsection}{\thesection.\arabic{subsection}} % 1.1, 1.2...
\renewcommand{\thesubsubsection}{\thesubsection.\arabic{subsubsection}} % 1.1.1, 1.1.2...
\let\OldTextField\TextField
\renewcommand{\TextField}[2][]{%
  \raisebox{-0.3ex}{\OldTextField[height=.85em,  bordercolor={1 1 1}, backgroundcolor={1 1 1},#1]{#2}}%
}
\renewcommand{\baselinestretch}{1.5}  % 1.5 es el valor estándar, pero puedes aumentarlo a 2, 2.5, etc.
\renewcommand{\rmdefault}{phv}  % Cambia la fuente de texto a Helvetica
\fontsize{12}{15}\selectfont  % Establece el tamaño de la fuente y la altura de línea
\onecolumn
% Configuración del pie de página
\pagestyle{fancy}
\fancyhf{} % Limpia cabeceras y pies de página
\fancyfoot[C]{\thepage} % Centra el número de página en el pie
% Eliminar líneas en cabecera y pie
\renewcommand{\headrulewidth}{0pt} % Sin línea en la cabecera
\renewcommand{\footrulewidth}{0pt} % Sin línea en el pie
%%%%%%%%%%%%%%%%%%%%%%%%%%%%%%%%%%%%%%%%%%%%%%%
\author{\TextField[name=Proyecto,width=16cm]{}}
\date{\today}

%%%%%%%%%%%%%%%%%%%%%%%%%%%%%%%%%%%%%%%%%%%%%%%
\begin{document}
%%%%%%%%%%%%%%%%%%%%%%%%%%%%%%%%%%%%%%%%%%%%%%%

\justifying
\begin{Form}
\maketitle
\vspace{10cm}
%%%%%%%%%%%%%%%%%%%%%%%%%%%%%%%%%%%%%%%%%%%%%%%

\begin{table}[h!]
    \centering
    \begin{tabular}{|p{2cm}|p{8cm}|}
        \hline
        Técnico: &     \TextField[name=Tecnico,width=6cm]{} 
        \\
        \hline
        Organización: &     \TextField[name=Organizacion,width=6cm]{} 
        \\
        \hline
        NIF\/NIE: &    \TextField[name = NIF,width=6cm]{} 
        \\ 
        \hline
        Fecha: &    \TextField[name = Fecha,width=6cm]{} 
        \\ 
        \hline
        Firma: &     
        \\ &
        \\ &
        \\ &
        \\ \hline
    \end{tabular}
\end{table}
%%%%%%%%%%%%%%%%%%%%%%%%%%%%%%%%%%%%%%%%%%%%%%%
\newpage
\resumen
\tableofcontents
\listoftables
\listoffigures

\newpage
% \newpage
% % %%%%%%%%%%%%%%%%%%%%%%%%%%%%%%%%%%%%%%%%%%%%%%%
% % %%%%%%%%%%%%%%%%%%%%%%%%%%%%%%%%%%%%%%%%%%%%%%%
% % %%%%%%%%%%%%%%%%%%%%%%%%%%%%%%%%%%%%%%%%%%%%%%%
% % %%%%%%%%%%%%%%%%%%%%%%%%%%%%%%%%%%%%%%%%%%%%%%%
% % %%%%%%%%%%%%%%%%%%%%%%%%%%%%%%%%%%%%%%%%%%%%%%%






\section{pruebas de grafico y tabla con una lista de datos}




% Datos no numéricos
\pgfplotstableread[col sep=comma]{
    Nombre dd dasdAS Asdas,Color,Forma
    Manzana,Rojo,{\TextField[name=Tecnico,width=6cm]{}}
    Plátano,Amarillo,Alargada
    Uva,Morado,Pequeña
    Limón,Verde,Ovalada
    Naranja,Naranja,Redonda
}\datosNoNumericos

\pgfplotstableread[col sep=comma]{
    x,y
    1,.2
    2,.3
    3,.5
    4,.7
    5,.11
}\misdatos


% Crear el gráfico
\begin{figure}[ht]
    \centering
    \begin{tikzpicture}[scale=.63]
        \begin{axis}[
            xlabel={Eje X},
            ylabel={Eje Y},
            grid=both,
            xtick=\empty, % No mostrar los valores del eje X
            ytick=\empty  % No mostrar los valores del eje Y
        ]
            \addplot+[ybar] table[x=x, y=y] {\misdatos};
            \addplot+ {1/sqrt(2*pi)*exp(-x^2/2)};
                  % Spline usando to[out=angle1, in=angle2]
        \end{axis}
    \end{tikzpicture}
    \caption{Una figura de ejemplo} % Título de la figura
\end{figure}




\begin{figure}[H]
    \centering
\begin{tikzpicture}[scale=.5, every node/.style={font=\small}]
    % Ejes
    \draw[] (0,0) -- (10,0) node[right, align=center] {i};
    \draw[] (0,-2) -- (0,4) node[above]{VAN};

    \draw[thick, blue] (.5,4) to[out=-50, in=150] (4,0);
    \draw[thick, blue] (4,0) to[out=-30, in=170] (9,-1);

    % Punto TIR
    \node[red] at (4,0) {\textbullet}; % Punto donde VAN = 0
    \node[below] at (4,0) {TIR};
\end{tikzpicture}
\caption{Curva de TIR: Punto de intersección donde VAN = 0.}
\label{fig:tir}
\end{figure}


% Tabla con datos no numéricos
\begin{table}[ht]
    \centering
    \caption{Una Tabla de ejemplo}
    \pgfplotstabletypeset[
        col sep=comma,
        every head row/.style={before row=\toprule, after row=\midrule},
        every last row/.style={after row=\bottomrule},
        string type % Indica que los datos son cadenas de texto
    ]{\datosNoNumericos}
\end{table}







\section{5.1.1 Parte inicial (véase capítulo 6 y tabla 1). La parte inicial debe contener los elementos siguientes, en el orden
que se indica a continuación:}

\begin{itemize}

    \item a) primera y segunda páginas de la cubierta, si son necesarias [6.1];
    \item b) portada [6.2]1);
    \item c) resumen [6.3];
    \item d) índice [6.4];
    \item e) glosario de signos, símbolos, unidades, abreviaturas, acrónimos o términos [6.5];
    \item f) prefacio, si es necesario [6.6].
\end{itemize}

\section{5.1.2 Cuerpo del informe (véase capítulo 7 y tabla 1). El cuerpo del informe debe constar de las siguientes par-
tes, en el orden que se indica a continuación:}

\begin{itemize}

    \item a) introducción [7.1];
    \item b) núcleo del informe con ilustraciones esenciales y tablas [7.2];
    \item c) conclusiones y recomendaciones [7.3];
    \item d) agradecimientos, si los hubiere [7.4];
    \item e) lista de referencias [7.5].
\end{itemize}
\section{5.1.3 Anexos (véase capítulo 8 y tabla 1). Los anexos se consideran separadamente de la parte final, debido a que,
aunque no siempre se requieren, pueden formar parte esencial de algunos informes.
}


\section{5.1.4 Parte final (véase capítulo 9 y tabla 1). La parte final debe contener los siguientes elementos en el orden
que se indica a continuación:}

\begin{itemize}

    \item a) hoja de datos del documento [9.1]1);
    \item b) lista de distribución y disponibilidad (fuentes y condiciones), si se requiere [9.2];
    \item c) cubierta posterior (páginas 3 y 4 de la cubierta), si se requiere [9.3].
\end{itemize}












































\section{Referencias}
\begin{itemize}
    \item UNE 50135:1996

\end{itemize}

\end{Form}
\end{document}
