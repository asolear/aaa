
\input{../../../assets/settings/newcommand.tex}

\documentclass[a4paper,12pt]{article}

\usepackage[utf8]{inputenc}
\usepackage[spanish]{babel}
% hay que usar para que no den error las flechas de draw, pero antes del tikz
% \shorthandoff{<} % Desactiva el uso de < como comando
% \shorthandoff{>} % Desactiva el uso de < como comando
\usepackage{graphicx}
\usepackage{titlesec}
\usepackage{geometry}
\usepackage{fancyhdr}
\usepackage[pdftex,pdfencoding=auto]{hyperref}
\usepackage{setspace}
\usepackage{tikz} % Para diagramas
\usepackage{amsmath} % For mathematical equations
\usepackage{colortbl} % Paquete para colores en tablas
\usepackage{pgfplots}
\usepackage{caption}
\usepackage{pgffor}  % Paquete para bucles
\usepackage{fancyhdr} % Para personalizar encabezados y pies de página
\usepackage{pgfplotstable}
\usepackage{booktabs}
\usepackage[T1]{fontenc}    % Soporte para caracteres con acentos
\usepackage{textcomp}       % Soporte adicional para símbolos
\usepackage{circuitikz}
\usepackage{glossaries} % Paquete para glosarios
\usepackage{forest}
\usepackage{ulem}
\usepackage{pythontex}

% 
\let\OldTextField\TextField
\renewcommand{\TextField}[2][]{%
  \raisebox{-0.1ex}{\OldTextField[height=.95em,  bordercolor={1 1 1}, backgroundcolor={1 1 1},#1]{#2}}%
}
\geometry{left=3cm, right=2.5cm, top=3cm, bottom=2.5cm}
\titleformat{\section}{\normalfont\Large\bfseries}{\thesection}{1em}{}
\titleformat{\subsection}{\normalfont\large\bfseries}{\thesubsection}{1em}{}
\pagestyle{fancy}
\fancyhf{}
\fancyhead[L]{}
\fancyfoot[C]{\thepage}
\setstretch{1.5}


\begin{document}
\begin{Form}

\graphicspath{{../../../assets/imgs}}



\begin{titlepage}
    \centering
    {\scshape\LARGE  \par}
    \vspace{1cm}
    {\Huge\bfseries  Recarga Ultrarrápida de Vehículo Eléctrico: Características\par}
    \vspace{2cm}
    {\Large Informe Técnico \par} 
    \vspace{1cm}
    % \includegraphics[width=0.4\textwidth]{logo.png}\par
    \vfill
    Autor: \TextField[name=Tecnico,width=6cm,default=Juan Pérez]{} \par
    Organización: \TextField[name=Organizacion,width=6cm,default=CalcAE]{} \par
    NIF: \TextField[name=NIF,width=6cm,default=12345678]{} \par
    Fecha:  NIF: \TextField[name=Fecha,width=6cm,default=\today]{} \par
\end{titlepage}
% Configuración del glosario



\begin{abstract}
La Regulación de Infraestructura para Combustibles Alternativos (\textbf{AFIR}, por sus siglas en inglés) es una iniciativa legislativa de la Unión Europea destinada a establecer un marco normativo para el desarrollo y la implementación de infraestructuras que soporten combustibles alternativos. Esto incluye la infraestructura necesaria para la recarga de vehículos eléctricos, así como para otros tipos de combustibles alternativos como el hidrógeno, el gas natural licuado (GNL), y el gas natural comprimido (GNC).
\end{abstract}


\tableofcontents

\section*{Glosario de términos}

\begin{description}
  \item[LaTeX] Sistema de preparación de documentos basado en texto, utilizado para crear documentos de alta calidad tipográfica.
  \item[PDF] Formato de archivo para documentos que preserva el formato y es independiente del software, hardware o sistema operativo utilizado.
  \item[UNE] Normas técnicas desarrolladas por la Asociación Española de Normalización y Certificación.
  \item[HTML] Lenguaje de marcado utilizado para la creación de páginas web.
  \item[CSS] Lenguaje utilizado para describir la presentación de un documento escrito en HTML o XML.
  \item[XML] Lenguaje de marcado que define reglas para la codificación de documentos en un formato legible por humanos y máquinas.
\end{description}


\newpage

\section{Introducción}
En esta sección se describe el propósito del documento y su alcance. También se puede incluir información general y antecedentes necesarios para entender el contenido.

\section{Objeto y Campo de Aplicación}
Definir de forma clara y concisa el propósito principal del documento y los límites de su aplicación.

\section{Normas y Referencias Aplicables}
Enumerar las normas, reglamentos y documentos de referencia utilizados.


\section{Metodología}
Describir el método o procedimiento seguido. Esto debe estar en concordancia con los estándares aplicables.

\section{Resultados}
Exponer los resultados obtenidos. Utiliza tablas o gráficos si es necesario.

\section{Conclusiones y Recomendaciones}
Presentar las conclusiones principales y sugerir posibles acciones o estudios futuros.

\section{Referencias}
Libro: PETTERSEN, Sverre. Introduction to Meteorology. New York, MacGraw Hill, 1941: pp. 200-210.


\appendix
\section{Anexos}
En esta sección se incluyen tablas, gráficos, cálculos o documentos adicionales que complementan el contenido principal.

\subsection{Ilustraciones o tablas suplementarias.}
\pgfplotstableread[col sep=comma]{
    x,y
    1,.2
    2,.3
    3,.5
    4,.7
    5,.11
}\misdatos

\begin{figure}[ht]
    \centering
    \begin{tikzpicture}[scale=.63]
        \begin{axis}[
            xlabel={Eje X},
            ylabel={Eje Y},
            grid=both,
            xtick=\empty, % No mostrar los valores del eje X
            ytick=\empty  % No mostrar los valores del eje Y
        ]
            \addplot+[ybar] table[x=x, y=y] {\misdatos};
            \addplot+ {1/sqrt(2*pi)*exp(-x^2/2)};
                  % Spline usando to[out=angle1, in=angle2]
        \end{axis}
    \end{tikzpicture}
    \caption{Una figura de ejemplo} % Título de la figura
\end{figure}


\pgfplotstableread[col sep=comma]{
    Nombre dd dasdAS Asdas,Color,Forma
    Manzana,Rojo,{\TextField[name=Tecnico,width=2cm,default=Juan Pérez]{}}
    Plátano,Amarillo,Alargada
    Uva,Morado,Pequeña
    Limón,Verde,Ovalada
    Naranja,Naranja,Redonda
}\datosNoNumericos

\begin{table}[ht]
    \centering
    \caption{Una Tabla de ejemplo}
    \pgfplotstabletypeset[
        col sep=comma,
        every head row/.style={before row=\toprule, after row=\midrule},
        every last row/.style={after row=\bottomrule},
        string type % Indica que los datos son cadenas de texto
    ]{\datosNoNumericos}
\end{table}

\begin{figure}[h]
    \centering
    \includegraphics[width=0.8\linewidth]{fvresidencial.jpg} % Ajusta la ruta y el tamaño según tus necesidades
    \caption{Emplazamiento geográfico.}
    \label{fig:etiqueta}
\end{figure}

\subsection{Descripción de equipos, técnicas o programas de ordenador..}







\newpage
\section*{Bibliografía}
\begin{itemize}
    \item \href{https://www.ree.es/es/clientes/generador/gestion-medidas-electricas/consulta-perfiles-de-consumo}{REE. Consulta los perfiles de consumo (TBD)}
    \item Norma UNE XXXX: Año. Título de la norma.
\end{itemize}





\section{Características de la Recarga Ultrarrápida}

Potencia de Carga:

Alta Potencia: Las estaciones de carga ultrarrápida suelen ofrecer potencias de 150 kW a 350 kW o más. Esta alta potencia permite cargar las baterías de los VE de manera mucho más rápida.

Tiempo de Carga:

Carga Rápida: Puede cargar un VE en un 80% en unos 15-30 minutos, dependiendo de la capacidad de la batería y la potencia de la estación de carga.

Infraestructura:

Equipos Especializados: Requiere infraestructura avanzada con equipos de alta capacidad para gestionar la energía necesaria.

Ubicación Estratégica: Estas estaciones suelen estar ubicadas en autopistas, estaciones de servicio, y áreas de descanso para facilitar la carga durante viajes largos.

Compatibilidad:

Conectores Estándar: Utiliza conectores estándar como CCS (Combined Charging System) en Europa y Estados Unidos, CHAdeMO en Japón, y GB/T en China.

Costos:

Costo por kWh: Generalmente, el costo por kWh es más alto en estaciones ultrarrápidas

\end{Form}
\end{document}


