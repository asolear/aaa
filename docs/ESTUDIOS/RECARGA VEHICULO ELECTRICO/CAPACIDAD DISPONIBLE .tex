\input{../../../assets/settings/newcommand.tex}

\documentclass[a4paper,12pt]{article}

\usepackage[utf8]{inputenc}
\usepackage[spanish]{babel}
% hay que usar para que no den error las flechas de draw, pero antes del tikz
% \shorthandoff{<} % Desactiva el uso de < como comando
% \shorthandoff{>} % Desactiva el uso de < como comando
\usepackage{graphicx}
\usepackage{titlesec}
\usepackage{geometry}
\usepackage{fancyhdr}
\usepackage[pdftex,pdfencoding=auto]{hyperref}
\usepackage{setspace}
\usepackage{tikz} % Para diagramas
\usepackage{amsmath} % For mathematical equations
\usepackage{colortbl} % Paquete para colores en tablas
\usepackage{pgfplots}
\usepackage{caption}
\usepackage{pgffor}  % Paquete para bucles
\usepackage{fancyhdr} % Para personalizar encabezados y pies de página
\usepackage{pgfplotstable}
\usepackage{booktabs}
\usepackage[T1]{fontenc}    % Soporte para caracteres con acentos
\usepackage{textcomp}       % Soporte adicional para símbolos
\usepackage{circuitikz}
\usepackage{glossaries} % Paquete para glosarios
\usepackage{forest}
\usepackage{ulem}
\usepackage{pythontex}

% 
\let\OldTextField\TextField
\renewcommand{\TextField}[2][]{%
  \raisebox{-0.1ex}{\OldTextField[height=.95em,  bordercolor={1 1 1}, backgroundcolor={1 1 1},#1]{#2}}%
}
\geometry{left=3cm, right=2.5cm, top=3cm, bottom=2.5cm}
\titleformat{\section}{\normalfont\Large\bfseries}{\thesection}{1em}{}
\titleformat{\subsection}{\normalfont\large\bfseries}{\thesubsection}{1em}{}
\pagestyle{fancy}
\fancyhf{}
\fancyhead[L]{}
\fancyfoot[C]{\thepage}
\setstretch{1.5}
\graphicspath{{../../../assets/imgs}}


\begin{document}
\begin{Form}



\begin{titlepage}
    \centering
    {\scshape\LARGE  \par}
    \vspace{1cm}
    {\Huge\bfseries Informe Técnico\par}
    \vspace{2cm}
    {\Large DETERMINACIÓN DE LA CAPACIDAD DISPONIBLE POR UN CONSUMIDOR
    DOMÉSTICO PARA REALIZAR LA RECARGA DEL VE SIN AMPLIAR LA POTENCIA\par}
    \vspace{1cm}
    % \includegraphics[width=0.4\textwidth]{logo.png}\par
    \vfill
    Autor: \TextField[name=Tecnico,width=6cm,default=Juan Pérez]{} \par
    Organización: \TextField[name=Organizacion,width=6cm,default=CalcAE]{} \par
    NIF: \TextField[name=NIF,width=6cm,default=12345678]{} \par
    Fecha:  NIF: \TextField[name=Fecha,width=6cm,default=\today]{} \par
\end{titlepage}
% Configuración del glosario

\begin{abstract}
  El texto del resumen debe estar de acuerdo con la Norma UNE 50103. En síntesis, debe ser tan informativo como lo
  permita la naturaleza del documento, para que los lectores puedan decidir si es necesario leer el documento completo;
  debe definir el objetivo, métodos, resultados y conclusiones presentadas en el documento original, bien en ese orden,
  o destacando inicialmente los resultados y conclusiones; debe constituir un texto completo, para que sea inteligible sin
  necesidad de referirse al documento. Debe ser conciso sin ser oscuro, reteniendo la información básica y el carácter
  del documento original. Los resúmenes de la mayoría de los informes deben tener menos de 250 palabras y en ningún
  caso más de 500; Deben estar escritos en un solo párrafo; emplear normalmente frases completas, verbos en forma
  activa y con tercera persona. No se deben utilizar figuras y símbolos, tales como tablas cortas y fórmulas, más que
  cuando no haya ninguna alternativa aceptable.\end{abstract}

\tableofcontents

\section*{Glosario de términos}

\begin{description}
  \item[LaTeX] Sistema de preparación de documentos basado en texto, utilizado para crear documentos de alta calidad tipográfica.
  \item[PDF] Formato de archivo para documentos que preserva el formato y es independiente del software, hardware o sistema operativo utilizado.
  \item[UNE] Normas técnicas desarrolladas por la Asociación Española de Normalización y Certificación.
  \item[HTML] Lenguaje de marcado utilizado para la creación de páginas web.
  \item[CSS] Lenguaje utilizado para describir la presentación de un documento escrito en HTML o XML.
  \item[XML] Lenguaje de marcado que define reglas para la codificación de documentos en un formato legible por humanos y máquinas.
\end{description}


\newpage

\section{Introducción}
En esta sección se describe el propósito del documento y su alcance. También se puede incluir información general y antecedentes necesarios para entender el contenido.

\section{Objeto y Campo de Aplicación}
Definir de forma clara y concisa el propósito principal del documento y los límites de su aplicación.

\section{Normas y Referencias Aplicables}
Enumerar las normas, reglamentos y documentos de referencia utilizados.


\section{Metodología}
Describir el método o procedimiento seguido. Esto debe estar en concordancia con los estándares aplicables.

\section{Resultados}
Exponer los resultados obtenidos. Utiliza tablas o gráficos si es necesario.

\section{Conclusiones y Recomendaciones}
Presentar las conclusiones principales y sugerir posibles acciones o estudios futuros.

\section{Referencias}
Libro: PETTERSEN, Sverre. Introduction to Meteorology. New York, MacGraw Hill, 1941: pp. 200-210.


\appendix
\section{Anexos}
En esta sección se incluyen tablas, gráficos, cálculos o documentos adicionales que complementan el contenido principal.

\subsection{Ilustraciones o tablas suplementarias.}
\pgfplotstableread[col sep=comma]{
    x,y
    1,.2
    2,.3
    3,.5
    4,.7
    5,.11
}\misdatos

\begin{figure}[ht]
    \centering
    \begin{tikzpicture}[scale=.63]
        \begin{axis}[
            xlabel={Eje X},
            ylabel={Eje Y},
            grid=both,
            xtick=\empty, % No mostrar los valores del eje X
            ytick=\empty  % No mostrar los valores del eje Y
        ]
            \addplot+[ybar] table[x=x, y=y] {\misdatos};
            \addplot+ {1/sqrt(2*pi)*exp(-x^2/2)};
                  % Spline usando to[out=angle1, in=angle2]
        \end{axis}
    \end{tikzpicture}
    \caption{Una figura de ejemplo} % Título de la figura
\end{figure}


\pgfplotstableread[col sep=comma]{
    Nombre dd dasdAS Asdas,Color,Forma
    Manzana,Rojo,{\TextField[name=Tecnico,width=2cm,default=Juan Pérez]{}}
    Plátano,Amarillo,Alargada
    Uva,Morado,Pequeña
    Limón,Verde,Ovalada
    Naranja,Naranja,Redonda
}\datosNoNumericos

\begin{table}[ht]
    \centering
    \caption{Una Tabla de ejemplo}
    \pgfplotstabletypeset[
        col sep=comma,
        every head row/.style={before row=\toprule, after row=\midrule},
        every last row/.style={after row=\bottomrule},
        string type % Indica que los datos son cadenas de texto
    ]{\datosNoNumericos}
\end{table}


\subsection{Descripción de equipos, técnicas o programas de ordenador..}






El operador del sistema (Red Eléctrica de España), calcula y publica regularmente las medidas de la demanda del
sistema eléctrico peninsular y los perfiles finales de consumo. Gracias al proyecto perfila, estos perfiles de consumo
aplicables a los consumidores domésticos se han podido determinar con precisión.

En base a esta información, y con el objetivo de poder estimar de una manera razonable y robusta el margen de
capacidad libre o “hueco” que tendrían los consumidores domésticos para realizar la cargar nocturna del VE, se han
tomado los valores máximos para cada periodo horario del coeficiente de perfilado A publicado por REE durante el
año 2015. Estos valores, ajustados en base 100 para el valor máximo de dicho coeficiente horario, han sido
representados en la siguiente gráfica


\begin{figure}[h]
    \centering
    \includegraphics[width=0.8\linewidth]{fvresidencial.jpg} % Ajusta la ruta y el tamaño según tus necesidades
    \caption{Emplazamiento geográfico.}
    \label{fig:etiqueta}
  \end{figure}


De esta manera, se obtiene el ratio horario de uso de la capacidad disponible por un consumidor doméstico.
Suponiendo que los VE fuera programados para que iniciaran su carga a partir de la 1 de la mañana (hora de inicio
de la tarifa de acceso supervalle, que coincide además con los precios más bajos de la energía en el mercado), un
consumidor doméstico tendría disponible en un escenario de máxima demanda para esta hora, prácticamente el
50\% de su capacidad de punta para poder realizar esta recarga.

En caso el de que se comprobara que los VE conectados a los puntos de recarga de las viviendas no realizan en su
mayoría una recarga lenta a partir de esta hora, este coeficiente debería ser recalculado.


\end{Form}
\end{document}
