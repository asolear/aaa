\input{../../../assets/settings/newcommand.tex}

\documentclass[a4paper,12pt]{article}

\usepackage[utf8]{inputenc}
\usepackage[spanish]{babel}
% hay que usar para que no den error las flechas de draw, pero antes del tikz
% \shorthandoff{<} % Desactiva el uso de < como comando
% \shorthandoff{>} % Desactiva el uso de < como comando
\usepackage{graphicx}
\usepackage{titlesec}
\usepackage{geometry}
\usepackage{fancyhdr}
\usepackage[pdftex,pdfencoding=auto]{hyperref}
\usepackage{setspace}
\usepackage{tikz} % Para diagramas
\usepackage{amsmath} % For mathematical equations
\usepackage{colortbl} % Paquete para colores en tablas
\usepackage{pgfplots}
\usepackage{caption}
\usepackage{pgffor}  % Paquete para bucles
\usepackage{fancyhdr} % Para personalizar encabezados y pies de página
\usepackage{pgfplotstable}
\usepackage{booktabs}
\usepackage[T1]{fontenc}    % Soporte para caracteres con acentos
\usepackage{textcomp}       % Soporte adicional para símbolos
\usepackage{circuitikz}

% 
\let\OldTextField\TextField
\renewcommand{\TextField}[2][]{%
  \raisebox{-0.1ex}{\OldTextField[height=.95em,  bordercolor={1 1 1}, backgroundcolor={1 1 1},#1]{#2}}%
}
\geometry{left=3cm, right=2.5cm, top=3cm, bottom=2.5cm}
\titleformat{\section}{\normalfont\Large\bfseries}{\thesection}{1em}{}
\titleformat{\subsection}{\normalfont\large\bfseries}{\thesubsection}{1em}{}
\pagestyle{fancy}
\fancyhf{}
\fancyhead[L]{}
\fancyfoot[C]{\thepage}
\setstretch{1.5}
\begin{document}
\begin{Form}




\begin{titlepage}
    \centering
    {\scshape\LARGE CalcAE \par}
    \vspace{1cm}
    {\Huge\bfseries Informe Técnico\par}
    \vspace{2cm}
    {\Large Norma UNE XXXX: Año \par}
    \vspace{1cm}
    % \includegraphics[width=0.4\textwidth]{logo.png}\par
    \vfill
    Autor: \TextField[name=Tecnico,width=6cm,default=Juan Pérez]{} \par
    Organización: \TextField[name=Organizacion,width=6cm,default=CalcAE]{} \par
    NIF: \TextField[name=NIF,width=6cm,default=12345678]{} \par
    Fecha:  NIF: \TextField[name=Fecha,width=6cm,default=\today]{} \par
\end{titlepage}

\tableofcontents
\newpage

\section{Introducción}
En esta sección se describe el propósito del documento y su alcance. También se puede incluir información general y antecedentes necesarios para entender el contenido.

\section{Objeto y Campo de Aplicación}
Definir de forma clara y concisa el propósito principal del documento y los límites de su aplicación.

\section{Normas y Referencias Aplicables}
Enumerar las normas, reglamentos y documentos de referencia utilizados.

\section{Definiciones y Abreviaturas}
Listar y definir términos importantes o abreviaturas empleadas en el documento.

\section{Metodología}
Describir el método o procedimiento seguido. Esto debe estar en concordancia con los estándares aplicables.

\section{Resultados}
Exponer los resultados obtenidos. Utiliza tablas o gráficos si es necesario.

\section{Conclusiones y Recomendaciones}
Presentar las conclusiones principales y sugerir posibles acciones o estudios futuros.

\appendix
\section{Anexos}
En esta sección se incluyen tablas, gráficos, cálculos o documentos adicionales que complementan el contenido principal.


\subsection{graficos}

\pgfplotstableread[col sep=comma]{
    x,y
    1,.2
    2,.3
    3,.5
    4,.7
    5,.11
}\misdatos

\begin{figure}[ht]
    \centering
    \begin{tikzpicture}[scale=.63]
        \begin{axis}[
            xlabel={Eje X},
            ylabel={Eje Y},
            grid=both,
            xtick=\empty, % No mostrar los valores del eje X
            ytick=\empty  % No mostrar los valores del eje Y
        ]
            \addplot+[ybar] table[x=x, y=y] {\misdatos};
            \addplot+ {1/sqrt(2*pi)*exp(-x^2/2)};
                  % Spline usando to[out=angle1, in=angle2]
        \end{axis}
    \end{tikzpicture}
    \caption{Una figura de ejemplo} % Título de la figura
\end{figure}


\pgfplotstableread[col sep=comma]{
    Nombre dd dasdAS Asdas,Color,Forma
    Manzana,Rojo,{\TextField[name=Tecnico,width=2cm,default=Juan Pérez]{}}
    Plátano,Amarillo,Alargada
    Uva,Morado,Pequeña
    Limón,Verde,Ovalada
    Naranja,Naranja,Redonda
}\datosNoNumericos

\begin{table}[ht]
    \centering
    \caption{Una Tabla de ejemplo}
    \pgfplotstabletypeset[
        col sep=comma,
        every head row/.style={before row=\toprule, after row=\midrule},
        every last row/.style={after row=\bottomrule},
        string type % Indica que los datos son cadenas de texto
    ]{\datosNoNumericos}
\end{table}

\newpage
\section*{Bibliografía}
\begin{itemize}
    \item Título del libro/artículo/documento. Autor(es). Año.
    \item Norma UNE XXXX: Año. Título de la norma.
\end{itemize}




La distancia d, medida sobre la horizontal, entre filas de módulos o entre una fila y un obstáculo
de altura h que pueda proyectar sombras, se recomienda que sea tal que se garanticen al menos
4 horas de sol en torno al mediodía del solsticio de invierno.
En cualquier caso, d ha de ser como mínimo igual a $h \cdot k$,, siendo k un factor adimensional al que,
en este caso, se le asigna el valor $1/\tan(61°- latitud)$.
En la tabla pueden verse algunos valores significativos del factor k, en función de la latitud
del lugar.
%%%%%%%%%%%%%%%%%%%%%%%%%%%%%%%%%%%%%%%%%%%%%%%%%%
\begin{table}[h]
    \centering
    \begin{tabular}{|c|c|c|c|c|c|c|}
        \hline
        \textbf{Latitud} & \textbf{29\textdegree} & \textbf{37\textdegree} & \textbf{39\textdegree} & \textbf{41\textdegree} & \textbf{43\textdegree} & \textbf{45\textdegree} \\ \hline
        $k$ & 1,600 & 2,246 & 2,475 & 2,747 & 3,078 & 3,487 \\ \hline
    \end{tabular}
\end{table}
%%%%%%%%%%%%%%%%%%%%%%%%%%%%%%%%%%%%%%%%%%%%%%%%%%

Asimismo, la separación entre la parte posterior de una fila y el comienzo de la siguiente no será
inferior a $h \cdot k$, siendo en este caso h la diferencia de alturas entre la parte alta de una fila y la
parte baja de la posterior, efectuándose todas las medidas con relación al plano que contiene las
bases de los módulos.



%%%%%%%%%%%%%%%%%%%%%%%%%%%%%%%%%%%%%%%%%%%%%%%%%%
\shorthandoff{<} % Desactiva el uso de < como comando
\shorthandoff{>} % Desactiva el uso de < como comando
\begin{figure}[h]
    \centering
    \begin{tikzpicture}
        % Definir coordenadas y ángulos
        \def\d{3} % distancia horizontal
        \def\h{2} % altura
        \def\angle{120} % ángulo de inclinación de las placas
        \def\length{1.5} % longitud de las placas

       % Dibujar el suelo y plataformas con rayado
    %    \fill[pattern=north east lines] (-2,-0.5) rectangle (10,0);
    \fill[color=gray](0,0) -- (0,1) -- (\d,1) -- (\d,0) -- cycle;
    \fill[color=gray] (\d,0) -- (\d,.5) -- (\d*2,.5) -- (\d*2,0) -- cycle;
    \fill[color=gray] (\d+\d,0) -- (\d+\d,1.5) -- (\d+\d+\d,1.5) -- (\d+\d+\d,0) -- cycle;

        % Definir la geometría de una placa fotovoltaica
        \begin{scope}[shift={(0.5,1)}, rotate=\angle]
            \draw[thick, fill=gray!30] (0,0) rectangle (\length,0.2);
        \end{scope}

        % Duplicar y rotar la geometrí$ de la placa fotovoltaica
        \begin{scope}[shift={(\d+1,\h-1.5)}, rotate=\angle]
            \draw[thick, fill=gray!30] (0,0) rectangle (\length,0.2);
        \end{scope}

        % Dibujar líneas de medición y etiquetas
        \draw[<->,thick] (0.5,-.2) -- node[below,fill=white] {$d_2$} ++(\d-.5,0);
        \draw[<->,thick] (\d+1,-.2) -- node[below,fill=white] {$d_1$} ++(\d-1,0);
        
        % \draw[dashed] (\d+\d, .5) -- (\d+\d,.5);  % Línea auxiliar para mostrar h
        \draw[<->,thick] (\d+\d-.2,1.5) -- node[fill=white, left] {$h_1$} ++(0,-1);

        \draw[<->,thick] (\d-.2,1) -- node[fill=white, left] {$h_2$} ++(0,.8);

        % Etiquetas de las figuras
        \node at (5,-1.5) {Fig. 7};
    \end{tikzpicture}
\end{figure}
%%%%%%%%%%%%%%%%%%%%%%%%%%%%%%%%%%%%%%%%%%%%%%%%%%

Si los módulos se instalan sobre cubiertas inclinadas, en el caso de que el azimut de estos, el de
la cubierta, o el de ambos, difieran del valor cero apreciablemente, el cálculo de la distancia
entre filas deberá efectuarse mediante la ayuda de un programa de sombreado para casos
generales suficientemente fiable, a fin de que se cumplan las condiciones requeridas.


\end{Form}
\end{document}
