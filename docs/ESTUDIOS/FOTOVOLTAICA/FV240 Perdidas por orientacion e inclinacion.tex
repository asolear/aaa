\input{../../../assets/settings/newcommand.tex}

\documentclass[a4paper,12pt]{article}

\usepackage[utf8]{inputenc}
\usepackage[spanish]{babel}
% hay que usar para que no den error las flechas de draw, pero antes del tikz
% \shorthandoff{<} % Desactiva el uso de < como comando
% \shorthandoff{>} % Desactiva el uso de < como comando
\usepackage{graphicx}
\usepackage{titlesec}
\usepackage{geometry}
\usepackage{fancyhdr}
\usepackage[pdftex,pdfencoding=auto]{hyperref}
\usepackage{setspace}
\usepackage{tikz} % Para diagramas
\usepackage{amsmath} % For mathematical equations
\usepackage{colortbl} % Paquete para colores en tablas
\usepackage{pgfplots}
\usepackage{caption}
\usepackage{pgffor}  % Paquete para bucles
\usepackage{fancyhdr} % Para personalizar encabezados y pies de página
\usepackage{pgfplotstable}
\usepackage{booktabs}
\usepackage[T1]{fontenc}    % Soporte para caracteres con acentos
\usepackage{textcomp}       % Soporte adicional para símbolos
\usepackage{circuitikz}

% 
\let\OldTextField\TextField
\renewcommand{\TextField}[2][]{%
  \raisebox{-0.1ex}{\OldTextField[height=.95em,  bordercolor={1 1 1}, backgroundcolor={1 1 1},#1]{#2}}%
}
\geometry{left=3cm, right=2.5cm, top=3cm, bottom=2.5cm}
\titleformat{\section}{\normalfont\Large\bfseries}{\thesection}{1em}{}
\titleformat{\subsection}{\normalfont\large\bfseries}{\thesubsection}{1em}{}
\pagestyle{fancy}
\fancyhf{}
\fancyhead[L]{}
\fancyfoot[C]{\thepage}
\setstretch{1.5}
\begin{document}
\begin{Form}




\begin{titlepage}
    \centering
    {\scshape\LARGE CalcAE \par}
    \vspace{1cm}
    {\Huge\bfseries Informe Técnico\par}
    \vspace{2cm}
    {\Large Norma UNE XXXX: Año \par}
    \vspace{1cm}
    % \includegraphics[width=0.4\textwidth]{logo.png}\par
    \vfill
    Autor: \TextField[name=Tecnico,width=6cm,default=Juan Pérez]{} \par
    Organización: \TextField[name=Organizacion,width=6cm,default=CalcAE]{} \par
    NIF: \TextField[name=NIF,width=6cm,default=12345678]{} \par
    Fecha:  NIF: \TextField[name=Fecha,width=6cm,default=\today]{} \par
\end{titlepage}

\tableofcontents
\newpage

\section{Introducción}
En esta sección se describe el propósito del documento y su alcance. También se puede incluir información general y antecedentes necesarios para entender el contenido.

\section{Objeto y Campo de Aplicación}
Definir de forma clara y concisa el propósito principal del documento y los límites de su aplicación.

\section{Normas y Referencias Aplicables}
Enumerar las normas, reglamentos y documentos de referencia utilizados.

\section{Definiciones y Abreviaturas}
Listar y definir términos importantes o abreviaturas empleadas en el documento.

\section{Metodología}
Describir el método o procedimiento seguido. Esto debe estar en concordancia con los estándares aplicables.

\section{Resultados}
Exponer los resultados obtenidos. Utiliza tablas o gráficos si es necesario.

\section{Conclusiones y Recomendaciones}
Presentar las conclusiones principales y sugerir posibles acciones o estudios futuros.

\appendix
\section{Anexos}
En esta sección se incluyen tablas, gráficos, cálculos o documentos adicionales que complementan el contenido principal.


\subsection{graficos}

\pgfplotstableread[col sep=comma]{
    x,y
    1,.2
    2,.3
    3,.5
    4,.7
    5,.11
}\misdatos

\begin{figure}[ht]
    \centering
    \begin{tikzpicture}[scale=.63]
        \begin{axis}[
            xlabel={Eje X},
            ylabel={Eje Y},
            grid=both,
            xtick=\empty, % No mostrar los valores del eje X
            ytick=\empty  % No mostrar los valores del eje Y
        ]
            \addplot+[ybar] table[x=x, y=y] {\misdatos};
            \addplot+ {1/sqrt(2*pi)*exp(-x^2/2)};
                  % Spline usando to[out=angle1, in=angle2]
        \end{axis}
    \end{tikzpicture}
    \caption{Una figura de ejemplo} % Título de la figura
\end{figure}


\pgfplotstableread[col sep=comma]{
    Nombre dd dasdAS Asdas,Color,Forma
    Manzana,Rojo,{\TextField[name=Tecnico,width=2cm,default=Juan Pérez]{}}
    Plátano,Amarillo,Alargada
    Uva,Morado,Pequeña
    Limón,Verde,Ovalada
    Naranja,Naranja,Redonda
}\datosNoNumericos

\begin{table}[ht]
    \centering
    \caption{Una Tabla de ejemplo}
    \pgfplotstabletypeset[
        col sep=comma,
        every head row/.style={before row=\toprule, after row=\midrule},
        every last row/.style={after row=\bottomrule},
        string type % Indica que los datos son cadenas de texto
    ]{\datosNoNumericos}
\end{table}

\subsection{Perdidas por orientacion e inclinacion}

\begin{center}

\begin{tikzpicture}

    % Escalar todo el dibujo
    \begin{scope}[scale=0.5]
        \def\L{5} % Longitud de la recta
        \def\theta{30} % Ángulo de la recta en grados
    
        % Dibujar el suelo
        \draw[thick] (-1, 0) -- (6, 0) ;
        
        % Dibujar el módulo inclinado
        \draw[thick] (0, 0) -- ({\L*cos(\theta)}, {\L*sin(\theta)});
        
        % Dibujar el ángulo beta
        \draw[color=red] (1, 0) arc[start angle=0, end angle=\theta, radius=1];
        \node[color=red] at (1.3, -0.5) {$\beta$};
        
        % Añadir etiquetas y dimensiones
        \node[above] at (1.5, 1.5) {Perfil del módulo};
    \end{scope}
    
    \end{tikzpicture}



%%%%%%%%%%%%%%%%%%%%%%%%%%%%%%%%%%%%%%%%%%%%%%%%%%%%%%%%%%%%%%%%%%%%%%%%%%%%%
\shorthandoff{<} % Desactiva el uso de < como comando
\shorthandoff{>} % Desactiva el uso de < como comando
\begin{tikzpicture}

    % Escalar todo el dibujo
    \begin{scope}[scale=0.5]
    % Dibujar las flechas de dirección
    \draw[thick, ->] (0, 0) -- (0, 3) node[above] {N}; % Flecha hacia el Norte
    \draw[thick, ->] (0, 0) -- (0, -3) node[below] {S}; % Flecha hacia el Sur
    \draw[thick, ->] (0, 0) -- (3, 0) node[right] {E}; % Flecha hacia el Este
    \draw[thick, ->] (0, 0) -- (-3, 0) node[left] {O}; % Flecha hacia el Oeste
    
    % Dibujar el teclado inclinado
    \begin{scope}[rotate=45]
        \draw[thick] (-1.5, -0.75) rectangle (1.5, 0.75);
        \foreach \x in {-1.2, -0.8, -0.4, 0, 0.4, 0.8, 1.2} {
            \draw (\x, -0.65) -- (\x, 0.65);
        }
        \foreach \y in {-0.45, -0.15, 0.15, 0.45} {
            \draw (-1.4, \y) -- (1.4, \y);
        }
    \end{scope}
    
    % Dibujar el ángulo alpha
    \draw[->] (0,0) -- (-90+45:2);
    
    \draw[color=red]  (0, -1.5) arc[start angle=-90, end angle=-90+45, radius=1.5];
    \node[color=red]  at (0.3, -2) {$\alpha$};
    
    % Añadir etiquetas y dimensiones
    \node[below left] at (-2.5, -2.5) {Fig. 2};
\end{scope}

    \end{tikzpicture}
    %%%%%%%%%%%%%%%%%%%%%%%%%%%%%%%%%%%%%%%%%%%%%%%%%%%%%%%%%%%%%%%%%%%%%%%%%%%%%
\end{center}



\newpage
\section*{Bibliografía}
\begin{itemize}
    \item Título del libro/artículo/documento. Autor(es). Año.
    \item Norma UNE XXXX: Año. Título de la norma.
\end{itemize}




\end{Form}
\end{document}
