\input{../../../assets/settings/newcommand.tex}

\documentclass[a4paper,12pt]{article}

\usepackage[utf8]{inputenc}
\usepackage[spanish]{babel}
% hay que usar para que no den error las flechas de draw, pero antes del tikz
% \shorthandoff{<} % Desactiva el uso de < como comando
% \shorthandoff{>} % Desactiva el uso de < como comando
\usepackage{graphicx}
\usepackage{titlesec}
\usepackage{geometry}
\usepackage{fancyhdr}
\usepackage[pdftex,pdfencoding=auto]{hyperref}
\usepackage{setspace}
\usepackage{tikz} % Para diagramas
\usepackage{amsmath} % For mathematical equations
\usepackage{colortbl} % Paquete para colores en tablas
\usepackage{pgfplots}
\usepackage{caption}
\usepackage{pgffor}  % Paquete para bucles
\usepackage{fancyhdr} % Para personalizar encabezados y pies de página
\usepackage{pgfplotstable}
\usepackage{booktabs}
\usepackage[T1]{fontenc}    % Soporte para caracteres con acentos
\usepackage{textcomp}       % Soporte adicional para símbolos
\usepackage{circuitikz}
\usepackage{glossaries} % Paquete para glosarios
\usepackage{forest}
\usepackage{ulem}

% 
\let\OldTextField\TextField
\renewcommand{\TextField}[2][]{%
  \raisebox{-0.1ex}{\OldTextField[height=.95em,  bordercolor={1 1 1}, backgroundcolor={1 1 1},#1]{#2}}%
}
\geometry{left=3cm, right=2.5cm, top=3cm, bottom=2.5cm}
\titleformat{\section}{\normalfont\Large\bfseries}{\thesection}{1em}{}
\titleformat{\subsection}{\normalfont\large\bfseries}{\thesubsection}{1em}{}
\pagestyle{fancy}
\fancyhf{}
\fancyhead[L]{}
\fancyfoot[C]{\thepage}
\setstretch{1.5}

% Configuración del glosario
\makeglossaries

% Definición de términos
\newglossaryentry{une}{
    name=UNE,
    description={Norma Española de carácter técnico}
}
\newglossaryentry{latex}{
    name=LaTeX,
    description={Sistema de preparación de documentos basado en TeX}
}
\newglossaryentry{pdf}{
    name=PDF,
    description={Formato de Documento Portátil (Portable Document Format)}
}

\begin{document}
\begin{Form}




\begin{titlepage}
    \centering
    {\scshape\LARGE  \par}
    \vspace{1cm}
    {\Huge\bfseries Informe Técnico\par}
    \vspace{2cm}
    {\Large Norma UNE 50135: Año \par}
    \vspace{1cm}
    % \includegraphics[width=0.4\textwidth]{logo.png}\par
    \vfill
    Autor: \TextField[name=Tecnico,width=6cm,default=Juan Pérez]{} \par
    Organización: \TextField[name=Organizacion,width=6cm,default=CalcAE]{} \par
    NIF: \TextField[name=NIF,width=6cm,default=12345678]{} \par
    Fecha:  NIF: \TextField[name=Fecha,width=6cm,default=\today]{} \par
\end{titlepage}
% Configuración del glosario

\begin{abstract}
  El texto del resumen debe estar de acuerdo con la Norma UNE 50103. En síntesis, debe ser tan informativo como lo
  permita la naturaleza del documento, para que los lectores puedan decidir si es necesario leer el documento completo;
  debe definir el objetivo, métodos, resultados y conclusiones presentadas en el documento original, bien en ese orden,
  o destacando inicialmente los resultados y conclusiones; debe constituir un texto completo, para que sea inteligible sin
  necesidad de referirse al documento. Debe ser conciso sin ser oscuro, reteniendo la información básica y el carácter
  del documento original. Los resúmenes de la mayoría de los informes deben tener menos de 250 palabras y en ningún
  caso más de 500; Deben estar escritos en un solo párrafo; emplear normalmente frases completas, verbos en forma
  activa y con tercera persona. No se deben utilizar figuras y símbolos, tales como tablas cortas y fórmulas, más que
  cuando no haya ninguna alternativa aceptable.\end{abstract}

\tableofcontents

\section*{Glosario de términos}

\begin{description}
  \item[LaTeX] Sistema de preparación de documentos basado en texto, utilizado para crear documentos de alta calidad tipográfica.
  \item[PDF] Formato de archivo para documentos que preserva el formato y es independiente del software, hardware o sistema operativo utilizado.
  \item[UNE] Normas técnicas desarrolladas por la Asociación Española de Normalización y Certificación.
  \item[HTML] Lenguaje de marcado utilizado para la creación de páginas web.
  \item[CSS] Lenguaje utilizado para describir la presentación de un documento escrito en HTML o XML.
  \item[XML] Lenguaje de marcado que define reglas para la codificación de documentos en un formato legible por humanos y máquinas.
\end{description}


\newpage

\section{Introducción}
En esta sección se describe el propósito del documento y su alcance. También se puede incluir información general y antecedentes necesarios para entender el contenido.
En este documento se explican conceptos como \gls{une}, \gls{latex} y \gls{pdf}.

\section{Objeto y Campo de Aplicación}
Definir de forma clara y concisa el propósito principal del documento y los límites de su aplicación.

\section{Normas y Referencias Aplicables}
Enumerar las normas, reglamentos y documentos de referencia utilizados.


\section{Metodología}
Describir el método o procedimiento seguido. Esto debe estar en concordancia con los estándares aplicables.

\section{Resultados}
Exponer los resultados obtenidos. Utiliza tablas o gráficos si es necesario.

\section{Conclusiones y Recomendaciones}
Presentar las conclusiones principales y sugerir posibles acciones o estudios futuros.

\section{Referencias}
Libro: PETTERSEN, Sverre. Introduction to Meteorology. New York, MacGraw Hill, 1941: pp. 200-210.


\appendix
\section{Anexos}
En esta sección se incluyen tablas, gráficos, cálculos o documentos adicionales que complementan el contenido principal.

\subsection{Ilustraciones o tablas suplementarias.}
\pgfplotstableread[col sep=comma]{
    x,y
    1,.2
    2,.3
    3,.5
    4,.7
    5,.11
}\misdatos

\begin{figure}[ht]
    \centering
    \begin{tikzpicture}[scale=.63]
        \begin{axis}[
            xlabel={Eje X},
            ylabel={Eje Y},
            grid=both,
            xtick=\empty, % No mostrar los valores del eje X
            ytick=\empty  % No mostrar los valores del eje Y
        ]
            \addplot+[ybar] table[x=x, y=y] {\misdatos};
            \addplot+ {1/sqrt(2*pi)*exp(-x^2/2)};
                  % Spline usando to[out=angle1, in=angle2]
        \end{axis}
    \end{tikzpicture}
    \caption{Una figura de ejemplo} % Título de la figura
\end{figure}


\pgfplotstableread[col sep=comma]{
    Nombre dd dasdAS Asdas,Color,Forma
    Manzana,Rojo,{\TextField[name=Tecnico,width=2cm,default=Juan Pérez]{}}
    Plátano,Amarillo,Alargada
    Uva,Morado,Pequeña
    Limón,Verde,Ovalada
    Naranja,Naranja,Redonda
}\datosNoNumericos

\begin{table}[ht]
    \centering
    \caption{Una Tabla de ejemplo}
    \pgfplotstabletypeset[
        col sep=comma,
        every head row/.style={before row=\toprule, after row=\midrule},
        every last row/.style={after row=\bottomrule},
        string type % Indica que los datos son cadenas de texto
    ]{\datosNoNumericos}
\end{table}


\subsection{Descripción de equipos, técnicas o programas de ordenador..}







\newpage
\section*{Bibliografía}
\begin{itemize}
    \item Título del libro/artículo/documento. Autor(es). Año.
    \item Norma UNE XXXX: Año. Título de la norma.
\end{itemize}


\section{Distancia mínima entre filas de módulos  }
\footnote{    IDAE. 
Instalaciones de
Energía Solar Fotovoltaica.
Pliego de Condiciones Técnicas de
Instalaciones Conectadas a Red
PCT-C-REV - julio 2011}
%%%%%%%%%%%%%%%%%%%%%%%%%%%%%%%%%%%%%%%%%%%%%%%%%%%%%%%%%%%%%%%%%%%%%%%%%%%%%%%

\subsection{Cálculo de Secciones en una Instalación Fotovoltaica con Microinversores}

%%%%%%%%%%%%%%%%%%%%%%%%%%%%%%%%%%%%%%%%%%%%%%%%%%%%%%%%%%%%%%%%%%%%%%%%%%%%%
\begin{figure}[h]
    \centering

  
    \pgfmathsetmacro\pmpt{\placafvpmp/1000*\instalacionnpaneles}
    \pgfmathsetmacro\pacot{\inversorpni/1000*\instalacionninversores }
  
  
      \begin{forest}
      for tree={
          draw,
          rectangle,
          align=center,
          edge={ line width=2.5pt},
          l sep=2cm, % Longitud de los bordes
          }
          [\instalacionnpaneles Modulos FV  de   \placafvpmp W
          [\instalacionninversores  Inversores de \inversorpni W  ($\frac{DC}{AC}  \equiv 1.4$), edge label={node[midway,right] {\pmpt kW}}
          [, edge label={node[midway,right] {\pacot kW}}
          ]]
      ]
      \end{forest}
      
      \caption{Esquema Electrico Resumen DC}
  \end{figure}
%%%%%%%%%%%%%%%%%%%%%%%%%%%%%%%%%%%%%%%%%%%%%%%%%%%%%%%%%%%%%%%%%%%%%%%%%%%%%

% %%%%%%%%%%%%%%%%%%%%%%%%%%%%%%%%%%%%%%%%%%%%%%%%%%%%%%%%%%%%%%%%%%%%%%%%%%%%%
% \pgfmathsetmacro\vs{\inversorpxc*\placafvvoc}
% \pgfmathsetmacro\is{\inversorcxs*\placafvisc}

% \begin{tikzpicture}[scale=1, node distance=1cm]
%     \node[rectangle, draw] (I) at (6,1+\inversorcxs) {$INV_i$};

%     \foreach \seguidor in {1,...,\inversorsxi} {
%         \node[rectangle, draw] (I_\seguidor) at (4,1+\inversorcxs*\seguidor) {{$MPPT_{i,\seguidor}$}};
%         \foreach \circuito in {1,...,\inversorcxs} {
%             \node[rectangle, draw] (FV\seguidor\circuito) at (0,\circuito+\inversorcxs*\seguidor) {{$FV_{i,\seguidor,1,\circuito}$}};
%             \node[rectangle, draw] (FVn\seguidor\circuito) at (2,\circuito+\inversorcxs*\seguidor) {{$FV_{i,\seguidor,\circuito,\inversorpxc}$}};
%             \draw[dashed] (FV\seguidor\circuito) -- (FVn\seguidor\circuito) ;
%             \draw[] (FVn\seguidor\circuito) -- (I_\seguidor) ;
%         }
%          \draw[] (I_\seguidor) -- (I);
%     }

% \end{tikzpicture}
% %%%%%%%%%%%%%%%%%%%%%%%%%%%%%%%%%%%%%%%%%%%%%%%%%%%%%%%%%%%%%%%%%%%%%%%%%%%%%





%%%%%%%%%%%%%%%%%%%%%%%%%%%%%%%%%%%%%%%%%%%%%%%%%%%%%%%%%%%%%%%%%%%%%%%%%%%%%
Para calcular las secciones de los conductores en una instalación fotovoltaica con microinversores de 1 kW cada uno y una potencia total de 20 kW, se deben tener en cuenta varios factores, tales como la corriente máxima, la distancia de los conductores y las normativas locales. A continuación, se presentan los pasos generales para el cálculo tanto en corriente continua (CC) como en corriente alterna (CA).

\subsubsection{Datos Iniciales}
\begin{itemize}
    \item Potencia total: 20 kW
    \item Potencia de cada microinversor: 1 kW
    \item Número de microinversores: 20 (20 kW / 1 kW por microinversor)
    \item Cada microinversor tiene 2 MPPT (Maximum Power Point Tracker)
\end{itemize}

\subsubsection{Cálculo en Corriente Continua (CC)}

En el lado de corriente continua, se encuentran los paneles solares antes de los microinversores.

\uline{Paso 1: Determinar la Corriente de los Paneles Solares}

Supongamos que cada panel solar tiene una tensión de operación de 40 V y una corriente de 10 A (estos valores pueden variar dependiendo del panel específico).

\uline{Paso 2: Agrupación de Paneles}

Si cada microinversor tiene 2 MPPT, y cada MPPT maneja un conjunto de paneles, necesitamos conocer la configuración específica de los paneles conectados a cada MPPT. Suponiendo una configuración típica donde cada MPPT maneja una cadena de 5 paneles en serie, la tensión total por cadena sería \(5 \times 40 \text{ V} = 200 \text{ V}\).

Cada cadena de paneles maneja 10 A, y como hay dos cadenas por microinversor, cada microinversor manejaría 10 A por MPPT en CC.

\uline{Paso 3: Calcular la Sección del Conductor en CC}

Utilizando la fórmula:

$$
S = \frac{2 \cdot L \cdot I}{\sigma \cdot U_{\text{adm}}}
$$

donde:
\begin{itemize}
    \item \(S\) es la sección del conductor en mm²,
    \item \(L\) es la longitud del conductor en metros,
    \item \(I\) es la corriente en amperios (10 A en este caso),
    \item \(\sigma\) es la conductividad del material (para cobre es 56 S/m·mm²),
    \item $U_{adm}$ es la caída de tensión admisible (normalmente se toma 1-3\%).
\end{itemize}

Para una longitud de 30 metros y una caída de tensión admisible del 1\%, calculamos:

$$
S = \frac{2 \cdot 30 \cdot 10}{56 \cdot 0.01 \cdot 200} \approx 0.54 \text{ mm}^2
$$

Por normativa y seguridad, normalmente se sobredimensionan los conductores. En este caso, se podría utilizar una sección de 1.5 mm² o 2.5 mm², dependiendo de las regulaciones locales.

\subsubsection{Cálculo en Corriente Alterna (CA)}

En el lado de corriente alterna, se encuentran los microinversores conectados a la red.

\uline{Paso 1: Determinar la Corriente en el Lado de CA}

La potencia total es 20 kW, y suponiendo una tensión de red de 230 V (monofásica) o 400 V (trifásica), calculamos la corriente.

Para sistema trifásico:

$$
I = \frac{P}{\sqrt{3} \cdot V \cdot \cos\phi}
$$

donde:
\begin{itemize}
    \item \(P\) es la potencia total (20,000 W),
    \item \(V\) es la tensión (400 V),
    \item \(\cos\phi\) es el factor de potencia (suponiendo 0.95).
\end{itemize}

$$
I = \frac{20000}{\sqrt{3} \cdot 400 \cdot 0.95} \approx 30.4 \text{ A}
$$

\uline{Paso 2: Calcular la Sección del Conductor en CA}

Utilizando la fórmula similar a la de CC, pero adaptada a CA:

$$
S = \frac{2 \cdot L \cdot I}{\sigma \cdot U_{\text{adm} \cdot \cos\phi}}
$$

Para una longitud de 30 metros y una caída de tensión admisible del 1\%:

$$
S = \frac{2 \cdot 30 \cdot 30.4}{56 \cdot 0.01 \cdot 400 \cdot 0.95} \approx 2.9 \text{ mm}^2
$$

Por normativa y seguridad, se recomendaría utilizar una sección de 6 mm² o 10 mm², dependiendo de las regulaciones locales y las condiciones específicas de la instalación.

\subsubsection{Resumen}

\begin{itemize}
    \item \textbf{Sección en CC}: Aproximadamente 2.5 mm², pero usualmente se usa una sección mayor por seguridad (4 mm²).
    \item \textbf{Sección en CA}: Aproximadamente 6 mm², pero se recomienda 10 mm² por normativa y condiciones específicas.
\end{itemize}

Es importante consultar las normativas locales y, de ser necesario, ajustar estos cálculos considerando factores adicionales como la temperatura, el tipo de aislamiento del conductor y otros factores específicos de la instalación.



\end{Form}
\end{document}
