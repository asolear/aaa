\input{../../../assets/settings/newcommand.tex}

\documentclass[a4paper,12pt]{article}

\usepackage[utf8]{inputenc}
\usepackage[spanish]{babel}
\usepackage{graphicx}
\usepackage{titlesec}
\usepackage{geometry}
\usepackage{fancyhdr}
\usepackage[pdftex,pdfencoding=auto]{hyperref}
\usepackage{setspace}
\usepackage{tikz} % Para diagramas
\usepackage{amsmath} % For mathematical equations
\usepackage{colortbl} % Paquete para colores en tablas
\usepackage{pgfplots}
\usepackage{caption}
\usepackage{pgffor}  % Paquete para bucles
\usepackage{fancyhdr} % Para personalizar encabezados y pies de página
\usepackage{pgfplotstable}
\usepackage{booktabs}
\usepackage[T1]{fontenc}    % Soporte para caracteres con acentos
\usepackage{textcomp}       % Soporte adicional para símbolos
\usepackage{circuitikz}

% 
\let\OldTextField\TextField
\renewcommand{\TextField}[2][]{%
  \raisebox{-0.1ex}{\OldTextField[height=.95em,  bordercolor={1 1 1}, backgroundcolor={1 1 1},#1]{#2}}%
}
\geometry{left=3cm, right=2.5cm, top=3cm, bottom=2.5cm}
\titleformat{\section}{\normalfont\Large\bfseries}{\thesection}{1em}{}
\titleformat{\subsection}{\normalfont\large\bfseries}{\thesubsection}{1em}{}
\pagestyle{fancy}
\fancyhf{}
\fancyhead[L]{}
\fancyfoot[C]{\thepage}
\setstretch{1.5}
\begin{document}
\begin{Form}


\begin{titlepage}
    \centering
    {\scshape\LARGE CalcAE \par}
    \vspace{1cm}
    {\Huge\bfseries Informe Técnico\par}
    \vspace{2cm}
    {\Large Norma UNE XXXX: Año \par}
    \vspace{1cm}
    % \includegraphics[width=0.4\textwidth]{logo.png}\par
    \vfill
    Autor: \TextField[name=Tecnico,width=6cm,default=Juan Pérez]{} \par
    Organización: \TextField[name=Organizacion,width=6cm,default=CalcAE]{} \par
    NIF: \TextField[name=NIF,width=6cm,default=12345678]{} \par
    Fecha:  NIF: \TextField[name=Fecha,width=6cm,default=\today]{} \par
\end{titlepage}

\tableofcontents
\newpage

\section{Introducción}
En esta sección se describe el propósito del documento y su alcance. También se puede incluir información general y antecedentes necesarios para entender el contenido.

\section{Objeto y Campo de Aplicación}
Definir de forma clara y concisa el propósito principal del documento y los límites de su aplicación.

\section{Normas y Referencias Aplicables}
Enumerar las normas, reglamentos y documentos de referencia utilizados.

\section{Definiciones y Abreviaturas}
Listar y definir términos importantes o abreviaturas empleadas en el documento.

\section{Metodología}
Describir el método o procedimiento seguido. Esto debe estar en concordancia con los estándares aplicables.

\section{Resultados}
Exponer los resultados obtenidos. Utiliza tablas o gráficos si es necesario.

\section{Conclusiones y Recomendaciones}
Presentar las conclusiones principales y sugerir posibles acciones o estudios futuros.

\appendix
\section{Anexos}
En esta sección se incluyen tablas, gráficos, cálculos o documentos adicionales que complementan el contenido principal.


\subsection{graficos}

\pgfplotstableread[col sep=comma]{
    x,y
    1,.2
    2,.3
    3,.5
    4,.7
    5,.11
}\misdatos

\begin{figure}[ht]
    \centering
    \begin{tikzpicture}[scale=.63]
        \begin{axis}[
            xlabel={Eje X},
            ylabel={Eje Y},
            grid=both,
            xtick=\empty, % No mostrar los valores del eje X
            ytick=\empty  % No mostrar los valores del eje Y
        ]
            \addplot+[ybar] table[x=x, y=y] {\misdatos};
            \addplot+ {1/sqrt(2*pi)*exp(-x^2/2)};
                  % Spline usando to[out=angle1, in=angle2]
        \end{axis}
    \end{tikzpicture}
    \caption{Una figura de ejemplo} % Título de la figura
\end{figure}


\pgfplotstableread[col sep=comma]{
    Nombre dd dasdAS Asdas,Color,Forma
    Manzana,Rojo,{\TextField[name=Tecnico,width=2cm,default=Juan Pérez]{}}
    Plátano,Amarillo,Alargada
    Uva,Morado,Pequeña
    Limón,Verde,Ovalada
    Naranja,Naranja,Redonda
}\datosNoNumericos

\begin{table}[ht]
    \centering
    \caption{Una Tabla de ejemplo}
    \pgfplotstabletypeset[
        col sep=comma,
        every head row/.style={before row=\toprule, after row=\midrule},
        every last row/.style={after row=\bottomrule},
        string type % Indica que los datos son cadenas de texto
    ]{\datosNoNumericos}
\end{table}

\newpage
\section*{Bibliografía}
\begin{itemize}
    \item Título del libro/artículo/documento. Autor(es). Año.
    \item Norma UNE XXXX: Año. Título de la norma.
\end{itemize}




\subsection{Diseno del campo fotovoltaico }
\footnote{    IDAE. 
Instalaciones de
Energía Solar Fotovoltaica.
Pliego de Condiciones Técnicas de
Instalaciones Conectadas a Red
PCT-C-REV - julio 2011
}
%%%%%%%%%%%%%%%%%%%%%%%%%%%%%%%%%%%%%%%%%%%%%%%%%%%%%%%%%%%%%%%%%%%%%%%%%%%%%
\begin{figure}[h]
\centering
\begin{tikzpicture}[x=0.01cm, y=0.4cm]

% Puntos de control
\def\Np{14}
\def\isc{13.1}
\def\voc{43.8*\Np}
\def\vocm{.8*\voc}

\def\vm{41.83*\Np}
\def\im{12.23*\Np}

\def\vni{600}
\def\vai{200}
\def\voxi{1000}
\def\vomi{200}

\draw[ mark=*,blue] plot[smooth] coordinates {(0,\isc) (\voc * 0.8,\isc * 0.9) (\voc,0)}  ;
\draw[ mark=square,red] plot[smooth] coordinates {(0,\isc) (\vocm * 0.8,\isc * 0.9) (\vocm,0)} ;

% Ejes
\draw[->.] (-.2,0) -- (\voxi*1.05,0) node[right] {$V_{STRING}$};
\draw[->.] (-.2,-1) -- (\voxi*1.05,-1) node[right] {$V_{INV}$};
\draw[->.] (0,-.2) -- (0,\isc*1.1) node[above] {$I$};

\draw[dashed,->.]   (\vocm* 0.8,\isc*1.1) -- (\vocm* 0.8,.190) ;

% Etiquetas
\fill[] (\vocm* 0.8,\isc*.9) circle (2pt) node[below left,black] {$P_{mp,70^oC}$};
\fill[] (\vocm* 0.8,-.0) circle (2pt) node[above left] {$V_{mp,70^oC}$};

\fill[] (\voc,0) circle (2pt) node[above right] {$V_{oc,0^oC}$};

\fill[] (\vni,-1) circle (2pt) node[below right] {$V_{nom}$};
\fill[] (\vai,-1) circle (2pt) node[below left] {$V_{arranq}$};
\fill[] (\voxi,-1) circle (2pt) node[below left] {$V_{OPmax}$};
\fill[] (\vomi,-1) circle (2pt) node[below right] {$V_{OPminq}$};
\fill[] (0,\isc) circle (2pt) node[below right] {$I_{SC}$};

\end{tikzpicture}
\caption{Curva IV de la cadena segun la temperatura}
\end{figure}
%%%%%%%%%%%%%%%%%%%%%%%%%%%%%%%%%%%%%%%%%%%%%%%%%%%%%%%%%%%%%%%%%%%%%%%%%%%%%




\end{Form}
\end{document}
