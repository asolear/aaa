\input{../../../assets/settings/newcommand.tex}

\documentclass[a4paper,12pt]{article}

\usepackage[utf8]{inputenc}
\usepackage[spanish]{babel}
% hay que usar para que no den error las flechas de draw, pero antes del tikz
% \shorthandoff{<} % Desactiva el uso de < como comando
% \shorthandoff{>} % Desactiva el uso de < como comando
\usepackage{graphicx}
\usepackage{titlesec}
\usepackage{geometry}
\usepackage{fancyhdr}
\usepackage{setspace}
\usepackage{tikz} % Para diagramas
\usepackage{amsmath} % For mathematical equations
\usepackage{colortbl} % Paquete para colores en tablas
\usepackage{pgfplots}
\usepackage{caption}
\usepackage{pgffor}  % Paquete para bucles
\usepackage{fancyhdr} % Para personalizar encabezados y pies de página
\usepackage{pgfplotstable}
\usepackage{booktabs}
\usepackage[T1]{fontenc}    % Soporte para caracteres con acentos
\usepackage{textcomp}       % Soporte adicional para símbolos
\usepackage{circuitikz}
\usepackage{glossaries} % Paquete para glosarios
% 
\usepackage[pdftex,pdfencoding=auto]{hyperref}
\let\OldTextField\TextField
\renewcommand{\TextField}[2][]{%
  \raisebox{-0.1ex}{\OldTextField[height=.95em,  bordercolor={1 1 1}, backgroundcolor={1 1 1},#1]{#2}}%
}
\geometry{left=3cm, right=2.5cm, top=3cm, bottom=2.5cm}
\titleformat{\section}{\normalfont\Large\bfseries}{\thesection}{1em}{}
\titleformat{\subsection}{\normalfont\large\bfseries}{\thesubsection}{1em}{}
\pagestyle{fancy}
\fancyhf{}
\fancyhead[L]{}
\fancyfoot[C]{\thepage}
\setstretch{1.5}


\begin{document}
\begin{Form}




\begin{titlepage}
    \centering
    {\scshape\LARGE  \par}
    \vspace{1cm}
    {\Huge\bfseries Informe Técnico\par}
    \vspace{2cm}
    {\Large Análisis de seguridad estructural de
    las cubiertas. Cálculo de los contrapesos de las
    instalaciones fotovoltaicas \par}
    \vspace{1cm}
    % \includegraphics[width=0.4\textwidth]{logo.png}\par
    \vfill
    Autor: \TextField[name=Tecnico,width=6cm,default=Juan Pérez]{} \par
    Organización: \TextField[name=Organizacion,width=6cm,default=CalcAE]{} \par
    NIF: \TextField[name=NIF,width=6cm,default=12345678]{} \par
    Fecha:  NIF: \TextField[name=Fecha,width=6cm,default=\today]{} \par
\end{titlepage}
% Configuración del glosario

\begin{abstract}
  El texto del resumen debe estar de acuerdo con la Norma UNE 50103. En síntesis, debe ser tan informativo como lo
  permita la naturaleza del documento, para que los lectores puedan decidir si es necesario leer el documento completo;
  debe definir el objetivo, métodos, resultados y conclusiones presentadas en el documento original, bien en ese orden,
  o destacando inicialmente los resultados y conclusiones; debe constituir un texto completo, para que sea inteligible sin
  necesidad de referirse al documento. Debe ser conciso sin ser oscuro, reteniendo la información básica y el carácter
  del documento original. Los resúmenes de la mayoría de los informes deben tener menos de 250 palabras y en ningún
  caso más de 500; Deben estar escritos en un solo párrafo; emplear normalmente frases completas, verbos en forma
  activa y con tercera persona. No se deben utilizar figuras y símbolos, tales como tablas cortas y fórmulas, más que
  cuando no haya ninguna alternativa aceptable.\end{abstract}

\tableofcontents

\section*{Glosario de términos}

\begin{description}
  \item[LaTeX] Sistema de preparación de documentos basado en texto, utilizado para crear documentos de alta calidad tipográfica.
  \item[PDF] Formato de archivo para documentos que preserva el formato y es independiente del software, hardware o sistema operativo utilizado.
  \item[UNE] Normas técnicas desarrolladas por la Asociación Española de Normalización y Certificación.
  \item[HTML] Lenguaje de marcado utilizado para la creación de páginas web.
  \item[CSS] Lenguaje utilizado para describir la presentación de un documento escrito en HTML o XML.
  \item[XML] Lenguaje de marcado que define reglas para la codificación de documentos en un formato legible por humanos y máquinas.
\end{description}


\newpage

\section{Introducción}
En esta sección se describe el propósito del documento y su alcance. También se puede incluir información general y antecedentes necesarios para entender el contenido.

\section{Objeto y Campo de Aplicación}
Definir de forma clara y concisa el propósito principal del documento y los límites de su aplicación.

\section{Normas y Referencias Aplicables}
Enumerar las normas, reglamentos y documentos de referencia utilizados.


\section{Metodología}
Describir el método o procedimiento seguido. Esto debe estar en concordancia con los estándares aplicables.

\section{Resultados}
Exponer los resultados obtenidos. Utiliza tablas o gráficos si es necesario.

\section{Conclusiones y Recomendaciones}
Presentar las conclusiones principales y sugerir posibles acciones o estudios futuros.

\section{Referencias}
Libro: PETTERSEN, Sverre. Introduction to Meteorology. New York, MacGraw Hill, 1941: pp. 200-210.


\appendix
\section{Anexos}
En esta sección se incluyen tablas, gráficos, cálculos o documentos adicionales que complementan el contenido principal.

\subsection{Ilustraciones o tablas suplementarias.}
\pgfplotstableread[col sep=comma]{
    x,y
    1,.2
    2,.3
    3,.5
    4,.7
    5,.11
}\misdatos

\begin{figure}[ht]
    \centering
    \begin{tikzpicture}[scale=.63]
        \begin{axis}[
            xlabel={Eje X},
            ylabel={Eje Y},
            grid=both,
            xtick=\empty, % No mostrar los valores del eje X
            ytick=\empty  % No mostrar los valores del eje Y
        ]
            \addplot+[ybar] table[x=x, y=y] {\misdatos};
            \addplot+ {1/sqrt(2*pi)*exp(-x^2/2)};
                  % Spline usando to[out=angle1, in=angle2]
        \end{axis}
    \end{tikzpicture}
    \caption{Una figura de ejemplo} % Título de la figura
\end{figure}


\pgfplotstableread[col sep=comma]{
    Nombre dd dasdAS Asdas,Color,Forma
    Manzana,Rojo,{\TextField[name=Tecnico,width=2cm,default=Juan Pérez]{}}
    Plátano,Amarillo,Alargada
    Uva,Morado,Pequeña
    Limón,Verde,Ovalada
    Naranja,Naranja,Redonda
}\datosNoNumericos

\begin{table}[ht]
    \centering
    \caption{Una Tabla de ejemplo}
    \pgfplotstabletypeset[
        col sep=comma,
        every head row/.style={before row=\toprule, after row=\midrule},
        every last row/.style={after row=\bottomrule},
        string type % Indica que los datos son cadenas de texto
    ]{\datosNoNumericos}
\end{table}


\subsection{Descripción de equipos, técnicas o programas de ordenador..}







\newpage
\section*{Bibliografía}
\begin{itemize}
    \item Título del libro/artículo/documento. Autor(es). Año.
    \item Norma UNE XXXX: Año. Título de la norma.
\end{itemize}



\
%%%%%%%%%%%%%%%%%%%%%%%%%%%%%%%%%%%%%%%%%%%%%%%%%%%%%%%%%%%%%%%%%%%%%%%%%%%%%%%
\subsection{Análisis de seguridad estructural}

% \subsection{Evaluación Inicial}
\subsubsection*{Introduccion}
    El análisis de seguridad estructural de las cubiertas con paneles fotovoltaicos es un proceso complejo que requiere una evaluación detallada de la estructura existente, un análisis riguroso de las nuevas cargas introducidas, y la posible implementación de refuerzos. Utilizando herramientas avanzadas de modelado y simulación, y asegurando el cumplimiento de las normativas locales, se puede garantizar que la adición de paneles fotovoltaicos sea segura y eficaz sin comprometer la integridad del edificio.

    % El presente documento, que no tiene carácter obligatorio, pretende dar un ejemplo del contenido mínimo de los Convenios CAE. Las partes pueden desarrollarlo y adaptarlo según sus necesidades.
\subsubsection*{Inspección Visual}
\begin{itemize}
    \item \textbf{Condición de la estructura existente:} Evaluar el estado actual del techo, incluyendo signos de desgaste, corrosión o daños estructurales.
    \item \textbf{Materiales de construcción:} Identificar los materiales de la cubierta y la estructura subyacente (madera, acero, hormigón, etc.).
\end{itemize}

\subsubsection*{Revisión Documental}
\begin{itemize}
    \item \textbf{Planos estructurales:} Revisar los planos originales del edificio para entender el diseño y las especificaciones estructurales.
    \item \textbf{Códigos y normativas:} Asegurarse de que el diseño cumpla con los códigos de construcción locales y las normativas específicas para instalaciones fotovoltaicas.
\end{itemize}

\subsection*{Análisis de Cargas}

\subsubsection*{Cargas Adicionales}
\begin{itemize}
    \item \textbf{Peso de los paneles fotovoltaicos:} Incluir el peso de los paneles, los marcos de soporte, y otros componentes del sistema.
    \item \textbf{Equipos adicionales:} Considerar el peso de inversores, cables, y otros equipos asociados.
\end{itemize}

\subsubsection*{Cargas Combinadas}
\begin{itemize}
    \item \textbf{Carga muerta:} Peso propio de la estructura del techo y cualquier acabado permanente.
    \item \textbf{Carga viva:} Peso de la nieve, mantenimiento y otras cargas temporales.
    \item \textbf{Carga de viento:} Evaluar cómo los paneles pueden afectar la carga de viento sobre la estructura. De acuerdo con ASCE (2022), los ensayos del túnel del viento permiten calcular los contrapesos necesarios en cada panel para asegurar la estabilidad del sistema \cite{idaepctcon}.
    \item \textbf{Carga sísmica:} En áreas propensas a terremotos, considerar cómo los paneles pueden influir en la respuesta sísmica del edificio.
\end{itemize}

\subsection*{Modelado y Simulación}

\subsubsection*{Modelado Estructural}
\begin{itemize}
    \item \textbf{Software de análisis estructural:} Utilizar programas como SAP2000, ETABS, o ANSYS para modelar la estructura del techo con los paneles fotovoltaicos.
    \item \textbf{Elementos finitos:} Crear un modelo de elementos finitos para una simulación precisa de las cargas y las respuestas estructurales.
\end{itemize}

\subsubsection*{Simulación de Cargas}
\begin{itemize}
    \item \textbf{Análisis estático y dinámico:} Realizar análisis tanto estáticos como dinámicos para entender cómo las cargas afectan la estructura.
    \item \textbf{Evaluación de puntos críticos:} Identificar las áreas de mayor esfuerzo y verificar que las tensiones y deformaciones estén dentro de los límites aceptables.
\end{itemize}

\subsection*{Refuerzo y Adaptación}

\subsubsection*{Necesidad de Refuerzos}
\begin{itemize}
    \item \textbf{Evaluación de capacidad:} Comparar la capacidad estructural existente con las nuevas demandas de carga.
    \item \textbf{Diseño de refuerzos:} Si es necesario, diseñar refuerzos estructurales como vigas adicionales, refuerzos de conexión, o refuerzos de la cubierta.
\end{itemize}

\subsubsection*{Implementación de Refuerzos}
\begin{itemize}
    \item \textbf{Materiales y técnicas:} Seleccionar materiales y técnicas de refuerzo adecuadas que no comprometan la funcionalidad de la cubierta ni interfieran con la instalación de los paneles.
    \item \textbf{Inspección y aprobación:} Realizar inspecciones durante y después de la implementación de los refuerzos para asegurar la conformidad con el diseño estructural.
\end{itemize}

\subsection*{Consideraciones Adicionales}

\subsubsection*{Mantenimiento y Monitoreo}
\begin{itemize}
    \item \textbf{Plan de mantenimiento:} Establecer un plan de mantenimiento regular para la estructura y los paneles fotovoltaicos.
    \item \textbf{Monitoreo estructural:} Implementar\cite{idaepctcon} sistemas de monitoreo para detectar cualquier cambio en el comportamiento estructural a lo largo del tiempo.
\end{itemize}

\subsubsection*{Seguridad y Acceso}
\begin{itemize}
    \item \textbf{Seguridad durante la instalación:} Garantizar que las prácticas de seguridad sean seguidas durante la instalación de los paneles.
    \item \textbf{Accesibilidad:} Asegurar que haya acceso adecuado para el mantenimiento regular de los paneles sin comprometer la seguridad estructural.
\end{itemize}



\end{Form}
\end{document}

