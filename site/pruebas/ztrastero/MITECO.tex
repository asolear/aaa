\documentclass[12pt]{article}
\usepackage[spanish]{babel}
\usepackage[utf8]{inputenc}
\usepackage[T1]{fontenc}     % Para caracteres latinos

\usepackage{geometry}
\usepackage{helvet}  % Helvetica (sans-serif)
\usepackage{array}
\usepackage{titlesec} % Para personalizar títulos de secciones
\usepackage[pdftex,pdfencoding=auto]{hyperref}
\usepackage{tabularx}
\usepackage{colortbl} % Paquete estándar para colorear celdas de tabla
\usepackage{ragged2e} % Para usar la justificación del texto
\usepackage{colortbl} % Paquete para colorear celdas de tablas
\usepackage{lipsum} % Para texto falso


% Redefinir \TextField para aplicar las opciones por defecto
\let\OldTextField\TextField
\renewcommand{\TextField}[2][]{%
  \OldTextField[height=1.1em, bordercolor={1 1 1}, borderwidth=0, backgroundcolor={1 1 1 0}, #1]{#2}%
}

\geometry{a4paper, margin=1in}

\setlength{\spaceskip}{4.8pt}
\renewcommand{\baselinestretch}{1.4}  % 1.5 es el valor estándar, pero puedes aumentarlo a 2, 2.5, etc.
\renewcommand{\rmdefault}{phv}  % Cambia la fuente de texto a Helvetica


% Configuración para centrar títulos en mayúsculas sin negrita
\titleformat{\section} % Cambia el formato de las secciones
  {\normalfont\filcenter} % Formato del texto (normal, centrado, tamaño grande)
  {} % Etiqueta (vacía para no mostrar números)
  {0pt} % Espaciado entre etiqueta y título
  {\MakeUppercase} % Convierte el título a mayúsculas


\begin{document}
\justifying

\begin{Form}


% Tabla que ocupa todo el ancho
\setlength{\tabcolsep}{10pt} % Espaciado entre columnas

\noindent
\begin{table}[h!]
    \begin{tabular}{|>{\centering\arraybackslash}p{4cm}|>{\centering\arraybackslash}p{11cm}|}
    \hline
    \textbf{Ficha} & \textbf{Ficha XXX010: Ficha imitación} \\ \hline
    Código & XXX010 \\ \hline
    Versión & V1.0 \\ \hline
    Sector & Agrario \\ \hline
    \end{tabular}
\end{table}

% \renewcommand{\baselinestretch}{1.5}  % Vuelve al interlineado por defecto


\section*{1. Ámbito de aplicación}
Instalación o sustitución de pantallas térmicas en invernaderos para cultivos agrícolas, plantas ornamentales o cultivos de flor. Se considera pantalla térmica al sistema pasivo de calefacción formado por hilos o filamentos de material poliéster o acrílico transparente o aluminizado, tejido o conformado en láminas continuas que actúan como barrera térmica entre el ambiente interior del invernadero y el exterior regulando el intercambio de calor.
\footnote{Esta es la nota al pie de página.}
\section*{2. Requisitos}
El invernadero debe disponer de instalación térmica (calefacción) para garantizar las condiciones de temperatura en el interior.
\section*{3. Cálculo del ahorro de energía}
El ahorro de energía se medirá en términos de energía final, expresada en kWh/año, de acuerdo con la siguiente fórmula:
\[
AE_{TOTAL} = S \cdot (K_i - K_p) \cdot (t_i - t_e) \cdot h
\]
Donde:
\begin{itemize}
    \item $S$: Superficie de la cubierta ($m^2$)
    \item $K_i$: Coeficiente global de pérdidas de calor por conducción-convección antes de la actuación ($kW/m^2 \cdot °C$)
    \item $K_p$: Coeficiente global de pérdidas de calor por conducción-convección después de la actuación ($kW/m^2 \cdot °C$)
    \item $t_i$: Temperatura del aire en el interior del invernadero ($°C$)
    \item $t_e$: Temperatura del aire en el exterior del invernadero ($°C$)
    \item $h$: Ciclo del cultivo (horas)
\end{itemize}

\section*{4. Resultado del cálculo}

\begin{table}[h!]
    \centering
    \begin{tabular}{|p{1cm}|p{1.5cm}|p{1.5cm}|p{1.5cm}|p{1.5cm}|p{1.5cm}|p{1.5cm}|p{1.5cm}|}
    \hline
    \textbf{S} & \textbf{K\textsubscript{a}} & \textbf{P\textsubscript{u}} & \textbf{Y\textsubscript{o}} & \textbf{Y\textsubscript{a}} & \textbf{h} & \textbf{AE\textsubscript{TOT}} & \textbf{Di} \\ \hline
    \TextField[name=S,width=1.5cm]{} & \TextField[name=K_a,width=1.5cm]{} & \TextField[name=P_u,width=1.5cm]{} & \TextField[name=Y_o,width=1.5cm]{} & \TextField[name=Y_a,width=1.5cm]{} & \TextField[name=h,width=1.5cm]{} & \TextField[name=AE_TOT,width=1.5cm]{} & \TextField[name=Di,width=1.5cm]{} \\ \hline
    \end{tabular}
    \caption{Tabla que ocupa todo el ancho de la página.}
\end{table}

\begin{table}[h!]
    \centering
    \begin{tabular}{|p{11cm}|p{4cm}|}
    \hline
    Duración indicativa de la actuación: & \hspace{2cm} años \\ \hline
    \end{tabular}
\end{table}

\begin{table}[h!]
    \centering
    \begin{tabular}{|p{5cm}|p{10cm}|}
    \hline
    Fecha inicio actuación: & \hspace{3cm} \\ \hline
    Fecha fin actuación: & \hspace{3cm} \\ \hline
    \end{tabular}
\end{table}

\begin{table}[h!]
    \centering
    \begin{tabular}{|p{5cm}|p{10cm}|}
    \hline
    Representante del solicitante: & \hspace{3cm} \\ \hline
    NIF/NIE: & \hspace{3cm} \\ \hline
    Firma electrónica: & \hspace{3cm} \\ \hline
    \end{tabular}
\end{table}

\section*{5. Documentos para la justificación de los ahorros de la actuación y su realización}
\begin{enumerate}
    \item Ficha cumplimentada y firmada por el representante legal del solicitante.
    \item Declaración responsable formalizada por el propietario inicial del ahorro de energía final referida a la solicitud y/u obtención de ayudas públicas.
    \item Facturas justificativas de la inversión realizada, que incluyan una descripción detallada de los elementos principales.
    \item Informe fotográfico antes y después de la actuación, que incluya las coordenadas geográficas.
    \item Declaración responsable donde figuren los cultivos y la duración de sus ciclos anuales expresados en horas.
    \item Declaración responsable formalizada por técnico competente, relativa al cálculo del coeficiente global de pérdidas de calor por conducción-convección antes y después de la actuación ($K_i$ y $K_p$).
\end{enumerate}
\end{Form}

\end{document}
