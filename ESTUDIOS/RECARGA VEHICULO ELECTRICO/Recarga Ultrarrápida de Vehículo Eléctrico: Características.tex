
% https://www.youtube.com/watch?v=RUxGAUID4JY

%%%%%%%%%%%%%%%%%%%%%%%%%%%%%%%%%%%%%%%%%%%%%%%%%%%%%%%%%%%%%%%%%%%%%%%%%%%%%%%
\documentclass{article}

\input{../../../assets/settings/newcommand.tex}
\input{../../../assets/settings/usepackage.tex}
\input{../../../assets/settings/quitarnumerossecciones.tex}

\usepackage{nopageno} % Paquete para desactivar la numeración de páginas
%%%%%%%%%%%%%%%%%%%%%%%%%%%%%%%%%%%%%%%%%%%%%%%%%%%%%%

\title{Recarga Ultrarrápida de Vehículo Eléctrico: Características }
\author{Kgnete}
\date{\today}

\begin{document}

\maketitle

\begin{abstract}


La Regulación de Infraestructura para Combustibles Alternativos (\textbf{AFIR}, por sus siglas en inglés) es una iniciativa legislativa de la Unión Europea destinada a establecer un marco normativo para el desarrollo y la implementación de infraestructuras que soporten combustibles alternativos. Esto incluye la infraestructura necesaria para la recarga de vehículos eléctricos, así como para otros tipos de combustibles alternativos como el hidrógeno, el gas natural licuado (GNL), y el gas natural comprimido (GNC).
\end{abstract}

\section{Características de la Recarga Ultrarrápida}

Potencia de Carga:

Alta Potencia: Las estaciones de carga ultrarrápida suelen ofrecer potencias de 150 kW a 350 kW o más. Esta alta potencia permite cargar las baterías de los VE de manera mucho más rápida.

Tiempo de Carga:

Carga Rápida: Puede cargar un VE en un 80% en unos 15-30 minutos, dependiendo de la capacidad de la batería y la potencia de la estación de carga.

Infraestructura:

Equipos Especializados: Requiere infraestructura avanzada con equipos de alta capacidad para gestionar la energía necesaria.

Ubicación Estratégica: Estas estaciones suelen estar ubicadas en autopistas, estaciones de servicio, y áreas de descanso para facilitar la carga durante viajes largos.

Compatibilidad:

Conectores Estándar: Utiliza conectores estándar como CCS (Combined Charging System) en Europa y Estados Unidos, CHAdeMO en Japón, y GB/T en China.

Costos:

Costo por kWh: Generalmente, el costo por kWh es más alto en estaciones ultrarrápidas


\end{document}
